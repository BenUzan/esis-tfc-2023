\documentclass[a4paper,12pt]{article}

\begin{document}
\section[Sujet]{SUJET}
\begin{center}
    Conception d’un système de gestion de fidélisation
    de la clientèle
    iIntégrant un système d’aide a la décision 
    %Conception d’un système d’aide à la décision 
    %de fidélisation de la clientèle.
    \textit{" Cas de la société de transport MULYKAP. "}        
\end{center}

\section[Context du sujet]{CONTEXTE DU SUJET}
Le bien fonder de ce travail de fin de cycle est de
vérifier la bonne acquisition des notions apprises tout
au long de notre parcours, ainsi que de leurs maitrises
dans le cycle de licence au sein de la filière de Management
des Systèmes d’Information.
% a ajouter ... up ...

\par Le choix de notre sujet se justifie par le fait que le marketing 
est un point essentiel du management.
\par La rentabilité de la masse d’information (pouvoir transformer
ce déluge d’information en valeur ajoutée), L’information étant
un gage de compétitivité %plein de blabla
\section[Problématique]{PROBLÉMATIQUES}
\subsection[Problèmes de recherche]{PROBLÈMES DE RECHERCHE}
La fidélisation du segment client dans les sociétés de transport
dans la province du Haut-Katanga 
\subsection[Questions de recherche]{QUESTIONS DE RECHERCHE}
Notre recherche s’articule autour des questions suivantes : 
\begin{itemize}
    \item Comment la société de transport MULYKAP gère-t-elle
    la relation client ?
    \item Comment MULYKAP fait-elle pour fidéliser sa clientèle ?
    \item Comment identifie-t-elle les clients pour une
    meilleure relation client ?
\end{itemize}

\section[Objectifs de recherche]{OBJECTIFS DE RECHERCHE}

\section[Hyposthèses générales]{HYPOTHÈSE GÉNÉRALES}

\section[Démarches méthodologiques]{DÉMARCHES MÉTHODOLOGIQUE}

\section[Etat de l'art]{ÉTAT DE L’ART}

\section[Résultats attenus]{RÉSULTAT ATTENDUS}

\section[Références bibliographiques]{RÉFÉRENCES BIBLIOGRAPHIQUE}
\end{document}