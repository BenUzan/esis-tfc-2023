\documentclass[a4paper,12pt,oneside]{book}
\usepackage[french]{babel}
\usepackage[T1]{fontenc}
\usepackage[left=3.5cm,right=2cm,top=3.5cm,bottom=2cm]{geometry}
\usepackage[sorting=none]{biblatex}
\usepackage{csquotes}
\usepackage{hyperref}

\addbibresource{proposition.bib}
\AtBeginDocument{\def\labelitemi{$\bullet$}}

\begin{document}

\frontmatter
    \chapter*{ÉPIGRAPHE}
    \enquote{\it Le travail de management d’une clientèle est à la vente ce
    que, pour l’agriculture, le travail de fertilisation et
    d’entretien des sols est à la récolte.}
    \begin{flushright}
        \it - Pascal PY
    \end{flushright}
    
    \chapter*{AVANT-PROPOS}
    Ce travail portant sur "La Conception d’un système de gestion de la fidélisation
    de la clientèle intégrant un système d’aide a la décision
    " cas de la société MULYKAP.
    \newline

    C’est dans le soucie de faciliter la fidélisation du segment client
    dans la Gestion de la Relation Clients, que nous avons été inspirer
    à réaliser ce travail pour apporter un plus a la société MULYKAP
    qui est notre cas d’étude et a toutes les entités qui détiennent des entreprises
    de transport désireuses d’innovation et de rentabiliser le 
    déluge d’information, de tirer profit de notre humble apport, fruit
    de nos recherches et de nos expériences acquise dans le domaine informatique.

\mainmatter

    \chapter*{INTRODUCTION}
        \section[Sujet]{Sujet}
        Conception d’un système de gestion de la fidélisation
        de la clientèle intégrant un système d’aide a la décision
        \textit{" Cas de la société de transport MULYKAP. "}

        \section[Contexte du sujet]{Contexte du sujet}
        %Doit entrer dans l'introduction
        La recherche scientifique anime aujourd’hui tout chercheur
        à pouvoir observer de manière
        plus particulière son environnement. Ce dernier étant
        sans cesse changeant, le chercheur se voit donc
        dans l’obligation de s’adapter aux différentes transformations
        survenant dans son environnement, ce qui n’est pas chose facile.
        Ainsi, suite à tous ces changements, le scientifique au cœur du
        développement rencontre plusieurs problèmes à résoudre en
        fonction de son domaine de recherche notamment, l’informatique,
        la médecine, l’architecture, l’économie, etc.
        \newline

        À l’exemple du domaine
        économique, la prise des décisions est devenue
        le point primordial pour une bonne gestion de
        l’entreprise, une bonne décision engendre l’efficience,
        c’est-à-dire la réalisation des objectifs
        poursuivis tout en produisant une valeur ajoutée.
        \newline

        La plupart des entreprises du monde disposent d’une masse de données plus ou
        moins considérable. Ces informations proviennent soit de sources internes (générées par
        leurs systèmes opérationnels au fil des activités journalières), ou bien de sources externes
        (web, partenaire, etc.). Cette surabondance de données, et l’impossibilité des systèmes
        opérationnels de les exploiter à des fins d’analyse conduit, inévitablement, l’entreprise à se
        tourner vers un nouvel informatique dite décisionnelle qui met l’accent sur la
        compréhension de l’environnement de l’entreprise et l’exploitation de ces données à bon
        escient.
        \newline

        En effet, les décideurs de l’entreprise ont besoin d’avoir une meilleure vision de
        leur environnement et de son évolution, ainsi, que des informations auxquelles ils peuvent
        se fier. Cela ne peut se faire qu’en mettant en place des indicateurs de performance clairs
        et pertinents permettant la sauvegarde, l’utilisation de la mémoire de l’entreprise et offrant
        à ses décideurs la possibilité de se projeter et de se reporter à ces indicateurs pour une bonne
        prise de décision.
        \newline

        Ainsi toutes les entreprises commerciales partagent aussi
        le plus souvent un certain nombre de souhaits : gagner
        du temps, prendre du recul par rapport aux urgences, obtenir une meilleure
        stabilité de leurs recettes, mieux organiser leur travail et obtenir un meilleur
        revenu. \cite*{Barouch2010}
        \newline

        Le choix de notre sujet se justifie par le fait que le marketing est un point essentiel du
        management. La rentabilité de la masse d’information (pouvoir transformer ce déluge
        d’information en valeur ajoutée), l’information étant un gage de compétitivité, surtout
        de nos jours, toute entreprise voulant prospérer se doit de prioriser la bonne gestion de
        cette dernière.

    \section[Problématique]{Problématique}
    La rédaction de tout travail scientifique implique au préalable une préoccupation.
    C’est dans le soucie de fidéliser le segment client de la société MULYKAP qui est une
    entreprise active dans le domaine du transport
        \subsection[Problèmes de recherche]{\textit{Problèmes de recherche}}
        La fidélisation du segment client dans les sociétés de transport
        dans la province du Haut-Katanga

        \subsection[Questions de recherche]{\textit{Questions de recherche}}
        De par le monde, les entreprises détenant une très grande part de marché
        se sentent le plus souvent à l’abri de toute concurrence dans leurs domaines d’activité
        grâce à la maitrise des rouages de leurs business ou
        d’une quelconque technologie. \cite*{Rouviere2010} Et de ce fait certain
        d’entre elles ne s’occupent plus tant que ça de leurs relations clients
        \newline

        C’est ainsi que nous prenons comme cas d’étude l’entreprise MULYKAP qui exerce 
        ses activités dans le domaine du transport et des hydrocarbures détient
        une position non négligeable sur le marché du transport (covoiturage).
        \newline

        L’entreprise MULYKAP rencontre des problèmes en ce qui concerne
        la gestion de leurs clientèles dont la perspective est de les
        fidéliser
        Notre recherche s’articule autour des questions suivantes :
        \newline 
            \begin{itemize}
                \item Comment la société de transport MULYKAP gère-t-elle
                la relation client ?
                \newline
                \item Comment MULYKAP fait-elle pour fidéliser sa clientèle ?
                \newline
                \item Comment identifie-t-elle les clients pour une
                meilleure relation client ?
                \newline
                \item De quelle manière obtient-elle le retour client ?
            \end{itemize}

    \section[Hypothèses générales]{Hypothèses générales}
    Après avoir réalisés des recherches préliminaires, nous pouvons émettre, au regard
    de notre problématique, les hypothèses suivantes :

    \section[Choix et interet du sujet]{Choix et intérêts du sujet}
        \subsection[Choix du sujet]{\textit{Choix du sujet}}
        Dans le cadre général, le choix porté sur ce sujet a été motivé par le souci d’aider
        la société MULYKAP de pouvoir fidéliser plus aisément le segment client à laide d’un
        système intégrant un système d’aide à la décision.
        \subsection[Interet du sujet]{\textit{Intérêts du sujet}}
            \begin{itemize}
                \item [-] Intérêt personnel : en tant qu’ingénieur en management 
                des systèmes d’information, serons heureux d’apporter une solution
                qui aidera les entreprises de transport de bien et de personne de fidéliser
                le segment client.
                \newline

                \item [-] Intérêt sociétal : la réalisation de ce travail aidera toutes les
                entreprises de transport des biens et des personnes dans leurs processus de
                fidélisation de leurs clientèles.
                \newline

                \item [-] Intérêt scientifique : dans le système LMD \footnote[1]{Licence-Master-Doctorat} tout étudiant
                a le devoir, à la fin de son cycle d’étude, d’élaborer un travail qui
                sanctionne son parcours. Le bien fonder du travail de fin de cycle
                est de donner l’occasion à l’étudiant de prouver la maitrise et la bonne acquisition
                des notions apprises tout au long de son parcours dans une filière données.
            \end{itemize}
    \section[Démarches méthodologiques]{Démarches méthodologiques}
        \subsection[Méthodes]{\textit{Méthodes}}
        Une méthode est une manière de conduire sa pensée, d’établir ou de démontrer une
        vérité suivant certains principes et avec un certain ordre.
        \newline

        Dans le cadre de notre travail, nous utiliserons la méthode UP (Unified Process).
        Le Processus Unifié est un processus de développement logiciel \enquote{itératif et incrémental,
        centré sur l’architecture, conduit par les cas d’utilisation et piloté par les risques.}\cite{Roques2008}
        \newline

        UML se définit comme un langage de modélisation graphique et textuel destiné à
        comprendre et décrire des besoins, spécifier et documenter des systèmes, esquisser des
        architectures logicielles, concevoir des solutions et communiquer des points de vue.\cite{RoqVall2007}
        \newline
        
        UML est le moyen graphique de garantir que \enquote{ce qui se conçoit et se programme
        bien s’énonce clairement.\footnote[2]{Hugues \textsc{Bersisni}, \textit{La programmation orientée objet}, 7e édition, Eyrolles, Paris, 2017, p. 222}}
        \subsection[Techniques]{\textit{Techniques}}
        Les techniques sont des mécanismes qui nous permettent de réaliser nos recherches
        scientifiques. Pour rendre notre travail facile à réaliser nous avons pris comme techniques : 
        \newline
        \begin{itemize}
            \item [-] L'interview : pour comprendre de façon simple notre travail, nous avons fait des
            descentes sur terrains, nous avons eu à interroger les techniciens et le manager
            informatiques, les professeurs informatiques, les collègues, les ainés scientifiques
            et ceux qui s’intéressent beaucoup plus à l’outil informatique.
            \newline
            \item [-] L'observation : nous nous sommes données beaucoup du temps à observer plus
            attentivement, pour comprendre de manière précise comment ça se passe exactement sur un ordinateur fonctionnel.
            \newline
            \item [-] La documentation : la lecture est le moyen efficace de voyager dans le monde
            de la connaissance. Comme un chercheur, pour parvenir à résoudre un problème
            dans la société, il nous faut une lecture consistante. Nous avons eu à utiliser les
            livres scientifiques, nous nous sommes rendus dans les bibliothèques, nous avons
            consulté des documents sur l’internet, nous avons utilisé différents médias afin de
            réunir toutes les informations dont nous avons besoin.
        \end{itemize}
        

    \section[Etat de l'art]{ÉTAT DE L’ART}
    Ce travail se basant sur des informations et une structure d’entreprise particulière, nous ne
    pouvons pas déclarer que ce dernier est une recherche originale, car d’autre chercheur ont eu
    à traiter du sujet relativement similaire à celui-ci.
    \newline

    Nous avons notamment BUYAMABA SUZE Ange, dans le cadre de son travail de fin de cycle
    qui parlait de : " MISE EN PLACE D’UN SYSTÈME INFORMATISE DE FIDÉLISATION DES CLIENTS BASE SUR LES
    DONNÉES D’UN SERVICE TRAITEUR". Ce travail \cite[prenote][postnote]{Buyamba2017}
    % Divergence et conc=vergence -- demarcation
    \section[Résultats attenus]{RÉSULTAT ATTENDUS}


\mainmatter

    \printbibliography

\end{document}