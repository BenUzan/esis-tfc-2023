\documentclass[a4paper,12pt,oneside]{book}
\usepackage[french]{babel}
\usepackage[T1]{fontenc}
\usepackage[left=3.5cm,right=2cm,top=3.5cm,bottom=2cm]{geometry}
\usepackage{biblatex}
\usepackage{csquotes}
\usepackage{hyperref}

\addbibresource{proposition.bib}
\AtBeginDocument{\def\labelitemi{$\bullet$}}

\begin{document}
\chapter*{INTRODUCTION}
\section[Sujet]{SUJET}
Conception d’un système de gestion de la fidélisation
de la clientèle intégrant un système d’aide a la décision 
%Conception d’un système d’aide à la décision 
%de fidélisation de la clientèle.
\textit{" Cas de la société de transport MULYKAP. "}
\section[Context du sujet]{CONTEXTE DU SUJET}
%Doit entrer dans l'introduction
La recherche scientifique anime aujourd’hui tout chercheur 
à pouvoir observer de manière
plus particulière son environnement. Ce dernier étant 
sans cesse changeant, le chercheur se voit donc
dans l’obligation de s’adapter aux différentes transformations
survenant dans son environnement, ce qui n’est pas chose facile.
Ainsi, suite à tous ces changements, le scientifique au cœur du
développement rencontre plusieurs problèmes à résoudre en 
fonction de son domaine de recherche notamment, l’informatique,
la médecine, l’architecture, l’économie, etc.
\newline

À l’exemple du domaine
économique, la prise des décisions est devenue
le point primordial pour une bonne gestion de
l’entreprise, une bonne décision engendre l’efficience,
c’est-à-dire la réalisation des objectifs
poursuivis tout en produisant une valeur ajoutée.
\newline

Toutes les entreprises commerciales partagent aussi
le plus souvent un certain nombre de souhaits : gagner
du temps, prendre du recul par rapport aux urgences, obtenir une meilleure
stabilité de leurs recettes, mieux organiser leur travail et obtenir un meilleur
revenu. \cite*{Barouch2010}
\newline

Le choix de notre sujet se justifie par le fait que le marketing 
est un point essentiel du management.
La rentabilité de la masse d’information (pouvoir transformer
ce déluge d’information en valeur ajoutée), L’information étant
un gage de compétitivité %plein de blabla

\section[Problématique]{PROBLÉMATIQUES}
La société MULYKAP est l’une des grandes entreprises
de la province du Haut-Katanga dans le domaine du transport
\subsection[Problèmes de recherche]{PROBLÈMES DE RECHERCHE}
La fidélisation du segment client dans les sociétés de transport
dans la province du Haut-Katanga

\subsection[Questions de recherche]{QUESTIONS DE RECHERCHE}
De par le monde, les entreprises détenant une très grande part de marché
se sentent le plus souvent à l’abri de toute concurrence dans leurs domaines d’activité
grâce à la maitrise des rouages de leurs business ou
d’une quelconque technologie. \cite*{Rouviere2010} 

C’est ainsi que nous prenons comme cas d’étude l’entreprise MULYKAP qui exerce ses activités dans le domaine 
du transport et des hydrocarbures détient
une position non négligeable sur le marché du transport (covoiturage). \cite{Rouviere2010}

L’entreprise MULYKAP rencontre des problèmes en ce qui concerne
la gestion de leurs clientèles dont la perspective est de les 
fidéliser 
Notre recherche s’articule autour des questions suivantes : 
\begin{itemize}
    \item Comment la société de transport MULYKAP gère-t-elle
    la relation client ?
    \item Comment MULYKAP fait-elle pour fidéliser sa clientèle ?
    \item Comment identifie-t-elle les clients pour une
    meilleure relation client ?
\end{itemize}

\section[Objectifs de recherche]{OBJECTIFS DE RECHERCHE}

\section[Hyposthèses générales]{HYPOTHÈSE GÉNÉRALES}

\section[Choix et interet du sujet]{CHOIX ET INTÉRÊT DU SUJET}
\subsection[Interet scientifique]{INTÉRÊT SCIENTIFIQUE}
Dans le système LMD \footnote[1]{Licence-Master-Doctorat}

\subsection[Interet personnel]{INTÉRÊT PERSONNEL}
Le bien fonder de ce travail de fin de cycle est de
vérifier la bonne acquisition des notions apprises tout
au long de notre parcours, ainsi que de leurs maitrises
dans le cycle de licence au sein de la filière de Management
des Systèmes d’Information.

\subsection[Interet societal]{INTÉRÊT SOCIÉTAL}

\section[Démarches méthodologiques]{DÉMARCHES MÉTHODOLOGIQUE}

\section[Etat de l'art]{ÉTAT DE L’ART}

\section[Résultats attenus]{RÉSULTAT ATTENDUS}

\printbibliography
\end{document}