\documentclass[a4paper,12pt,oneside]{book}
\usepackage[left=3.5cm,right=2cm,top=3.5cm,bottom=2cm]{geometry}
\usepackage[skip=10pt plus1pt, indent=40pt]{parskip}
\usepackage[utf8]{inputenc}
\usepackage[T1]{fontenc}
\usepackage[french]{babel}
\usepackage{newtxtext,newtxmath}
\usepackage[sorting=none]{biblatex}
\usepackage[acronym]{glossaries}
\usepackage{booktabs}
\usepackage{longtable}
\usepackage{hyperref}
\usepackage[labelfont=it, textfont=it, singlelinecheck=false]{caption}
\usepackage{amsmath}
\usepackage{graphicx}
\usepackage{float}
\usepackage{csquotes}
\usepackage{enumitem}
\usepackage[explicit]{titlesec}
\usepackage{sectsty}
\usepackage{pifont}
\usepackage{hologo}
\usepackage{setspace}
\usepackage{fancyhdr}

\pagestyle{headings}
\pagestyle{fancy}
\fancyhead[R]{\thepage}
\fancyhead[L]{}
\fancyfoot{} % clear footer fields
\fancyfoot[L]{\textit{TFC ESIS}}
\fancyfoot[R]{\textit{MSI 2023}}
\renewcommand\headrulewidth{0pt}
\renewcommand\footrulewidth{0pt}

%------------------------------------------------
%	Importation des fichiers
%------------------------------------------------
\addbibresource{main.bib}
\graphicspath{ {./images} }

%------------------------------------------------
%	Profondeur des sections et niveau de
%   consideration dans la TOC
%------------------------------------------------
\setcounter{secnumdepth}{4}
\setcounter{tocdepth}{2}

%------------------------------------------------
%	Parametres des tableaux
%------------------------------------------------
\setlength{\tabcolsep}{12pt}
\renewcommand{\arraystretch}{1}

%------------------------------------------------
%	Parametres du sectionnement grace a
%   sectsty
%------------------------------------------------
\sectionfont{\selectfont\large}
\subsectionfont{\selectfont\large\itshape}
\subsubsectionfont{\normalfont\large\itshape}
\paragraphfont{\normalfont\large\itshape}

%------------------------------------------------
%	Parametres du sectionnement grace a
%   titlesec
%------------------------------------------------
\titleformat{\chapter}[display]{\centering\selectfont\large\bfseries}{Chapitre \thechapter}{12pt}{\centering\MakeUppercase{#1}}
\titlespacing{\chapter}{10pt}{-40pt}{30pt}

%------------------------------------------------
%	Commandes de traduction
%------------------------------------------------
\addto\captionsfrench{\renewcommand{\listfigurename}{LISTE DES FIGURES}}
\addto\captionsfrench{\renewcommand{\listtablename}{LISTE DES TABLEAUX}}
\addto\captionsfrench{\renewcommand{\contentsname}{TABLE DES MATIÈRES}}
\addto\captionsfrench{\def\tablename{Tableau}}


%----------------------------------------------------------------------------------
%	Informations des pages de gardes
%----------------------------------------------------------------------------------
\newcommand{\HRule}{\rule{\linewidth}{0.25mm}} % Ligne horizontal leger
\newcommand{\HHRule}{\rule{300pt}{0.5mm}} % Ligne horizontal gras
\newcommand{\ATfc}{UZAN MUYUMBA Benjamin} % Auteur du TFC
\newcommand{\DTfc}{Prof. LISONGOMI BATIBONDA André} % Nom du directeur (Mr. Mme. Prof.)
\newcommand{\Filiere}{Management des Systèmes d’Information} % Filerer ou option
\newcommand{\TTfc}{MISE EN PLACE D’UN SYSTÈME DE FIDÉLISATION DE LA CLIENTÈLE
INTÉGRANT UN MODULE D’AIDE À LA DÉCISION DANS UNE ENTREPRISE DE TRANSPORT
DE BIENS ET DE PERSONNES\\\enquote{CAS DE LA SOCIÉTÉ NGOKAF TRANS}} % Sujet ou titre du TFC + cas
\newcommand{\MATfc}{Novembre 2023} % Mois et annees de defense du TFc

\newacronym{uml}{UML}{Unified Modeling Language}
\newacronym{up}{UP}{Unified Process}
\newacronym{bi}{BI}{Business Intelligence}
\newacronym{sad}{SAD}{Système d’Aide à la Décision}
\newacronym{si}{SI}{Système d’Information}
\newacronym{dss}{DSS}{Decision Support System}

\makenoidxglossaries

\setstretch{1.25}

\begin{document}
    \frontmatter
        \pagenumbering{Roman}
        %----------------------------------------------------------------------------------------
%	TITLE PAGE
%----------------------------------------------------------------------------------------

\begin{titlepage} % Suppresses displaying the page number on the title page and the subsequent page counts as page 1
	\center % Centre everything on the page
	
	%------------------------------------------------
	%	Headings
	%------------------------------------------------
	
	{\large ÉCOLE SUPÉRIEURE D’INFORMATIQUE SALAMA}\\ % Main heading such as the name of your university/college
	
	{\large République Démocratique du Congo}\\ % Major heading such as course name
	
	{\large Province du Haut-Katanga}\\ % Minor heading such as course title
	
	{\large Lubumbashi}\\ % Minor heading such as course title
	
	{\large \href{www.esisalama.org}{www.esisalama.org}}\\[0.25cm] % Minor heading such as course title

	\HHRule\\[1cm]

    \includegraphics[width=40mm]{images/logoesis.png}\\[0.5cm]

	%------------------------------------------------
	%	Title
	%------------------------------------------------
	
	\HRule\\[0.4cm]
	
	\textbf{MISE EN PLACE D’UN SYSTÈME DE FIDÉLISATION DE LA CLIENTÈLE
	INTÉGRANT UN MODULE D’AIDE À LA DÉCISION DANS UNE ENTREPRISE DE TRANSPORT
	DE BIENS ET DE PERSONNES\\\enquote{CAS DE LA SOCIÉTÉ NGOKAF TRANS}}
	% Title of your document
	
	\HRule\\[1.5cm]
	%------------------------------------------------
	%	Author(s)
	%------------------------------------------------
	\hfill
	\begin{minipage}{0.6\linewidth}
		\textit{Travail présenté et défendu en vue de l’obtention
		du grade d’ingénieur technicien en Management des
		Systèmes d’Information}\\
	\end{minipage}
	\\
	\hfill
	\begin{minipage}{0.6\textwidth}
		\textit{\textbf{Rédigé par : UZAN MUYUMBA Benjamin}}
		\\
		\textit{\textbf{Option  : Management des Systèmes d’Information}}
	\end{minipage}
	
	%------------------------------------------------
	%	Date
	%------------------------------------------------
	
	\vfill\vfill\vfill % Position the date 3/4 down the remaining page
	
	\textbf{Novembre 2023}
\end{titlepage}

\begin{titlepage} % Suppresses displaying the page number on the title page and the subsequent page counts as page 1
	\center % Centre everything on the page
	
	%------------------------------------------------
	%	Headings
	%------------------------------------------------
	
	{\large ÉCOLE SUPÉRIEURE D’INFORMATIQUE SALAMA}\\ % Main heading such as the name of your university/college
	
	{\large République Démocratique du Congo}\\ % Major heading such as course name
	
	{\large Province du Haut-Katanga}\\ % Minor heading such as course title
	
	{\large Lubumbashi}\\ % Minor heading such as course title
	
	{\large \href{www.esisalama.org}{www.esisalama.org}}\\[0.25cm] % Minor heading such as course title
	
	\HHRule\\[1cm]

    \includegraphics[width=40mm]{images/logoesis.png}\\[0.5cm]

	%------------------------------------------------
	%	Title
	%------------------------------------------------
	
	\HRule\\[0.4cm]
	
	\textbf{MISE EN PLACE D’UN SYSTÈME DE FIDÉLISATION DE LA CLIENTÈLE
	INTÉGRANT UN MODULE D’AIDE À LA DÉCISION DANS UNE ENTREPRISE DE TRANSPORT
	DE BIENS ET DE PERSONNES\\\enquote{CAS DE LA SOCIÉTÉ NGOKAF TRANS}}
	% Title of your document
	
	\HRule\\[1.5cm]
	%------------------------------------------------
	%	Author(s)
	%------------------------------------------------
	\hfill
	\begin{minipage}{0.6\linewidth}
		\textit{Travail présenté et défendu en vue de l’obtention
		du grade d’ingénieur technicien en Management des
		Systèmes d’Information}\\
	\end{minipage}
	\\
	\hfill
	\begin{minipage}{0.6\textwidth}
		\textit{\textbf{Rédigé par : UZAN MUYUMBA Benjamin}}
		\\
		\textit{\textbf{Option  : Management des Systèmes d’Information}}\\
	\end{minipage}
	\\
	\hfill
	\begin{minipage}{0.6\textwidth}
		\textit{\textbf{Directeur : Prof. LISONGOMI
		BATIPONDA André}}
	\end{minipage}
	
	%------------------------------------------------
	%	Date
	%------------------------------------------------
	
	\vfill\vfill\vfill % Position the date 3/4 down the remaining page
	
	\textbf{Novembre 2023}
\end{titlepage}

        \chapter*{ÉPIGRAPHE}
\addcontentsline{toc}{chapter}{ÉPIGRAPHE}
    \enquote{\it Le travail de management d’une clientèle est à la vente ce
    que, pour l’agriculture, le travail de fertilisation et
    d’entretien des sols est à la récolte.}
    \begin{flushright}
        \it - Pascal PY
    \end{flushright}

        \chapter*{IN MEMORIAM}
\addcontentsline{toc}{chapter}{IN MEMORIAM}
    \enquote{\it À mon feu père MUYUMBA LUTEYA Patrice et à ma très
    chère tante  TSHINYAMA MUTOMBO Sophie qui n’a cessée
    de me motiver par ses conseils.}

        \addcontentsline{toc}{chapter}{DÉDICACE}
\chapter*{DÉDICACE}
\thispagestyle{empty}
\clearpage
        \chapter*{REMERCIEMENTS}
\addcontentsline{toc}{chapter}{REMERCIEMENTS}

        \phantomsection
\addcontentsline{toc}{chapter}{LISTE DES FIGURES}
\listoffigures

        \phantomsection
\addcontentsline{toc}{chapter}{LISTE DES TABLEAUX}
\listoftables

        %\phantomsection
\addcontentsline{toc}{chapter}{LISTE DES EQUATIONS}
        \phantomsection
\addcontentsline{toc}{chapter}{LISTE DES ACRONYMES}
\printnoidxglossary[type=\acronymtype, title=LISTE DES ACRONYMES]

        %%%%%%%%%%%%%%%%%%%%%%%%%%%%%%%%%%%%%%
%% Faudra revoir le numero de pages %%
%%%%%%%%%%%%%%%%%%%%%%%%%%%%%%%%%%%%%%
\phantomsection
\addcontentsline{toc}{chapter}{TABLE DES MATIÈRES}
\tableofcontents



        \chapter*{AVANT-PROPOS}
\addcontentsline{toc}{chapter}{AVANT-PROPOS}
Ce travail portant sur la \enquote{Mise en place d’un système de
fidélisation de la clientèle intégrant un module d’aide à la
décision dans une entreprise de transport
de biens et de personnes} cas de la société Ngokaf Trans.
\par
C’est dans le soucie de faciliter la fidélisation du segment client
dans la Gestion de la relation clients, que nous avons été inspire
à réaliser ce travail pour apporter un plus à la société Ngokaf Trans
qui est notre cas d’étude et à toutes les entités qui détiennent des entreprises
de transport désireuses d’innovation et de rentabiliser le 
déluge d’information, de tirer profit de notre humble apport, fruit
de nos recherches et de nos expériences acquises dans le domaine informatique.
    \mainmatter
        \chapter*{INTRODUCTION GÉNÉRALE}
\addcontentsline{toc}{chapter}{INTRODUCTION GÉNÉRALE}
    %%%%%%%%%%%%%%%%%%%%%%%%%%%%%%%%%%%%%
    %% SECCTION A ABSOLUMENT SUPPRIMER %%
    %%%%%%%%%%%%%%%%%%%%%%%%%%%%%%%%%%%%%
    \section[Sujet]{Sujet}
    La présente étude porte sur la \enquote{Conception d’un système de gestion
    de la fidélisation de la clientèle intégrant un système d’aide à la décision}.
    À ce sujet, nous avons pris le cas de la société de transport TRANS NGOKAF.
    \section[Contexte du sujet]{Contexte du sujet}
    La recherche scientifique anime aujourd’hui tout chercheur
    à pouvoir observer de manière
    plus particulière son environnement. Ce dernier étant
    sans cesse changeant, le chercheur se voit donc
    dans l’obligation de s’adapter aux différentes transformations
    survenant dans son environnement, ce qui n’est pas chose facile.
    Ainsi, suite à tous ces changements, le scientifique au cœur du
    développement rencontre plusieurs problèmes à résoudre en
    fonction de son domaine de recherche notamment, l’informatique,
    la médecine, l’architecture, l’économie, etc.
    \newline

    À l’exemple du domaine
    économique, la prise des décisions est devenue
    le point primordial pour une bonne gestion de
    l’entreprise. Une bonne décision engendre l’efficience,
    c’est-à-dire la réalisation des objectifs
    poursuivis tout en produisant une valeur ajoutée.
    \newline

    La plupart des entreprises du monde disposent d’une masse de données plus ou
    moins considérable. Ces informations proviennent soit de sources internes (générées par
    leurs systèmes opérationnels au fil des activités journalières), ou bien de sources externes
    (web, partenaire, etc.). Cette surabondance de données, et l’impossibilité des systèmes
    opérationnels de les exploiter à des fins d’analyse conduit, inévitablement, l’entreprise à se
    tourner vers un nouvel informatique dite décisionnelle qui met l’accent sur la
    compréhension de l’environnement de l’entreprise et l’exploitation de ces données à bon
    escient.
    \newline

    En effet, les décideurs de l’entreprise ont besoin d’avoir une meilleure vision de
    leur environnement et de son évolution, ainsi, que des informations auxquelles ils peuvent
    se fier. Cela ne peut se faire qu’en mettant en place des indicateurs de performance clairs
    et pertinents permettant la sauvegarde, l’utilisation de la mémoire de l’entreprise et offrant
    à ses décideurs la possibilité de se projeter et de se reporter à ces indicateurs pour une bonne
    prise de décision.
    \newline

    Ainsi toutes les entreprises commerciales partagent aussi
    le plus souvent un certain nombre de souhaits : gagner
    du temps, prendre du recul par rapport aux urgences, obtenir une meilleure
    stabilité de leurs recettes, mieux organiser leur travail et obtenir un meilleur
    revenu. \cite*{Barouch2010}
    \newline

    Le choix de notre sujet se justifie par le fait que le marketing est un point essentiel du
    management. La rentabilité de la masse d’information (pouvoir transformer ce déluge
    d’information en valeur ajoutée), l’information étant un gage de compétitivité, surtout
    de nos jours, toute entreprise voulant prospérer se doit de prioriser la bonne gestion de
    cette dernière.
    \section[Problématique]{Problématique}
    La rédaction de tout travail scientifique implique au préalable une préoccupation.
    C’est dans le soucie de fidéliser le segment client de la société TRANS NGOKAF qui est une
    entreprise active dans le domaine du transport des biens et des personnes que nous
    réalisons ce travail.
    \newline
        
    De par le monde, les entreprises détenant une très grande part de marché
    se sentent le plus souvent à l’abri de toute concurrence dans leurs domaines d’activité
    grâce à la maitrise des rouages de leurs business ou
    d’une quelconque technologie. \cite*{Rouviere2010} Et de ce fait certain
    d’entre elles ne s’occupent plus tant que ça de leurs relations clients
    \newline

    C’est ainsi que nous prenons comme cas d’étude l’entreprise TRANS NGOKAF qui exerce 
    ses activités dans le domaine du transport, et qui détient
    une position non négligeable sur le marché du transport (covoiturage).
    \newline

    L’entreprise TRANS NGOKAF rencontre des problèmes en ce qui concerne
    la gestion de leurs clientèles dont la perspective est de les
    fidéliser.
    Notre recherche s’articule autour des questions suivantes :
    \newline 
        \begin{itemize}
            \item [\ding{226}] Comment la société de transport TRANS NGOKAF gère-t-elle
            la relation client ?
            \newline
            \item [\ding{226}] Comment TRANS NGOKAF fait-elle pour fidéliser sa clientèle ?
            \newline
            \item [\ding{226}] Comment identifie-t-elle les clients pour une
            meilleure relation client ?
            \newline
            \item [\ding{226}] De quelle manière obtient-elle le retour client ?
        \end{itemize}
    \section[Hypothèses générales]{Hypothèses générales}
    Après avoir réalisés des recherches préliminaires, nous pouvons émettre, au regard
    de notre problématique, les hypothèses suivantes : 
    \newline
    \begin{itemize}
        \item [\ding{226}] Une application web sera réalisée, ce qui permettra de collecter
        les informations de la clientèle et leurs retours concernant les prestations de l’entreprise.
        \newline
        \item [\ding{226}] Afin d’effectuer du marketing ciblé sur les clients, un outil d’aide à
        la décision sera intégrer à l’application.
        \newline
        \item [\ding{226}] Un moyen de fidélisation par campagne promotionnelle sera réaliser.
        % \item [-] Visualiser les ventes
        
    \end{itemize}
    \section[Choix et interet du sujet]{Choix et intérêts du sujet}
        \subsection[Choix du sujet]{\textit{Choix du sujet}}
        Dans le cadre général, le choix porté sur ce sujet a été motivé par le souci d’aider
        la société TRANS NGOKAF de pouvoir fidéliser plus aisément le segment client à laide d’un
        système intégrant un système d’aide à la décision.
        \subsection[Interet du sujet]{\textit{Intérêts du sujet}}
        L’intérêt du sujet se réfère à l’importance et à la pertinence du sujet choisi pour notre recherche.
        Un sujet intéressant est celui qui est pertinent pour les autres parties impliquées dans nos recherches.
        Il doit être capable de susciter l’intérêt et de motiver les recherches et la rédaction. C’est ainsi
        que nous nous en retiendrons trois :
         \newline 
            \begin{itemize}
                \item [\ding{226}] Intérêt personnel : en tant qu’ingénieur en management 
                des systèmes d’information, serons heureux d’apporter une solution
                qui aidera les entreprises de transport de bien et de personne de fidéliser
                le segment client.
                \newline

                \item [\ding{226}] Intérêt sociétal : la réalisation de ce travail aidera toutes les
                entreprises de transport des biens et des personnes dans leurs processus de
                fidélisation de leurs clientèles.
                \newline

                \item [\ding{226}] Intérêt scientifique : dans le système LMD \footnote[1]{Licence-Master-Doctorat} tout étudiant
                a le devoir, à la fin de son cycle d’étude, d’élaborer un travail qui
                sanctionne son parcours. Le bien fonder du travail de fin de cycle
                est de donner l’occasion à l’étudiant de prouver la maitrise et la bonne acquisition
                des notions apprises tout au long de son parcours dans une filière données.
            \end{itemize}
    \section[Démarches méthodologiques]{Démarches méthodologiques}
        \subsection[Méthodes]{\textit{Méthodes}}
        Une méthode est une manière de conduire sa pensée, d’établir ou de démontrer une
        vérité suivant certains principes et avec un certain ordre.
        \newline

        Dans le cadre de notre travail, nous utiliserons la méthode UP (Unified Process).
        Le Processus Unifié est un processus de développement logiciel \enquote{itératif et incrémental,
        centré sur l’architecture, conduit par les cas d’utilisation et piloté par les risques.} \cite{Roques2008}
        \newline

        UML se définit comme un langage de modélisation graphique et textuel destiné à
        comprendre et décrire des besoins, spécifier et documenter des systèmes, esquisser des
        architectures logicielles, concevoir des solutions et communiquer des points de vue. \cite{RoqVall2007}
        \newline
        
        UML est le moyen graphique de garantir que \enquote{ce qui se conçoit et se programme
        bien s’énonce clairement.\footnote[2]{Hugues \textsc{Bersisni}, \textit{La programmation orientée objet}, 7e édition, Eyrolles, Paris, 2017, p. 222}}
        \subsection[Techniques]{\textit{Techniques}}
        Les techniques sont des mécanismes qui nous permettent de réaliser nos recherches
        scientifiques. Pour rendre notre travail facile à réaliser nous avons pris comme techniques : 
        \newline
        \begin{itemize}
            \item [\ding{226}] L’interview : pour comprendre de façon simple notre travail, nous avons fait des
            descentes sur terrains, nous avons eu à interroger les techniciens et le manager
            informatiques, les professeurs informatiques, les collègues, les ainés scientifiques
            et ceux qui s’intéressent beaucoup plus à l’outil informatique.
            \newline
            \item [\ding{226}] L’observation : nous nous sommes données beaucoup du temps à observer plus
            attentivement, pour comprendre de manière précise comment ça se passe exactement sur un ordinateur fonctionnel.
            \newline
            \item [\ding{226}] La documentation : la lecture est le moyen efficace de voyager dans le monde
            de la connaissance. Comme un chercheur, pour parvenir à résoudre un problème
            dans la société, il nous faut une lecture consistante. Nous avons eu à utiliser les
            livres scientifiques, nous nous sommes rendus dans les bibliothèques, nous avons
            consulté des documents sur l’internet, nous avons utilisé différents médias afin de
            réunir toutes les informations dont nous avons besoin.
        \end{itemize}
    \section[Etat de l'art]{État de l’art}
    Ce travail se basant sur des informations et une structure d’entreprise particulière, nous ne
    pouvons pas déclarer que ce dernier est une recherche originale, car d’autre chercheur ont eu
    à traiter du sujet relativement similaire à celui-ci.
    \newline

    Nous avons notamment BUYAMABA SUZE Ange, dans le cadre de son travail de fin de cycle
    qui parlait de : \enquote{MISE EN PLACE D’UN SYSTÈME INFORMATISE DE FIDÉLISATION DES CLIENTS BASE SUR LES
    DONNÉES D’UN SERVICE TRAITEUR}. Ce travail base sur la conception d’un système informatisé de fidélisation des clients
    à aider la maison Excellence d’obtenir les retours clients dans la perspective d’une amélioration des services. \cite{Buyamba2017}
    \newline

    Cependant, il est impossible à la maison Excellence de savoir, grâce à des analyses et des visualisations, qui
    sont les meilleurs clients. Et sans oublier que sa recherche était concentré sur les services traiteurs. 
    \newline

    Mais en ce qui nous concerne nous rejoignions le chercheur cité ci-haut, dans le sens où nous traitons de manière
    générale d’une application de fidélisation, mais intégrant un système d’aide à la décision.
    % Divergence et conc=vergence -- demarcation
    \section[Délimitation du travail]{Délimitation du travail}
        Nos recherches ont été effectuer chez TRANS NGOKAF qui est une entreprise de transport
        située dans la ville de Lubumbashi, dans la province du Haut-Katanga en République Démocratique du Congo.
        \newline
        Notre Travail couvrira l’année 2023.
    \section[Subdivision du travail]{Subdivision du travail}
    Outre l’introduction générale et la conclusion générale, notre travail est subdivisé
    en trois chapitres :
    \newline
        \begin{itemize}
            \item [\ding{226}] Le premier chapitre : \enquote{cadre conceptuel et théorique}, présente une vue
            d’ensembles des concepts de base du sujet et quelques théories sur la méthodologie
            utilisée du travail.
            \newline
            \item [\ding{226}] Le deuxième chapitre : \enquote{analyse conceptuelle du système d’information}, 
            décrit l’analyse de l’architecture métier dans lequel nous allons présenter un système existant
            et le processus de l’organisation du travail ainsi que le futur système.
            \newline       
            \item [\ding{226}] Le troisième chapitre : \enquote{analyse et conception du système}, s’appuie sur la
            modélisation de notre système métier et est basé sur l’explication de différentes
            technologies utilisées pour l’implémentation ou développement de notre solution.            
        \end{itemize} 
    \section[Outils logiciels et équipements utilisés]{Outils logiciels et équipements utilisés}
    %%%%%%%%%%%%%
    %% A VENIR %%
    %%%%%%%%%%%%%
    Après avoir présenté le contexte et les enjeux de notre recherche, nous allons maintenant nous pencher sur
    les différents concepts, théories et méthodes utilisées.
        \renewcommand{\thechapter}{\Roman{chapter}}
        \chapter[CADRE CONCEPTUEL ET THÉORIQUE]{CADRE CONCEPTUEL ET THÉORIQUE}    
    \section{Introduction partielle}
    Actuellement, l’informatique est devenue l’outil privilégié de l’information dans
    différentes entreprises. D’où sa nécessité s’impose dans tous les domaines de gestion et la
    maitrise de l’outil informatique qui constitue la garantie d’une bonne gestion, d’un bon
    fonctionnement et donc une bonne rentabilité de cette dernière.
    \section[Cadre Conceptuel]{Cadre conceptuel}
    Le cadre conceptuel met en relation les concepts
    fondamentaux du travail de recherche. Dans cette
    section, nous allons aborder les théories existantes
    relatives à notre domaine d’étude comprenant trois
    aspects à savoir :
    \par
        \begin{enumerate}
            \item La gestion de la relation client
            \item La fidélisation de la clientèle
            \item Les outils d’écoute
            \item Le système décisionnel
        \end{enumerate} 
        \subsection[La gestion de la relation client]{\textit{La gestion de la relation client}}
            \subsubsection[Définition]{\textit{Définition}}
            La gestion de la relation client (CRM, customer
            relationship management en anglais) est la capacité
            à bâtir une relation profitable sur le long terme
            avec les meilleurs clients en capitalisant sur 
            l’ensemble des points de contacts par une allocation optimale
            des ressources. \cite*{Lefebure2005}
            \par
            L’objectif de la gestion de la relation client
            est de créer et d’entretenir une relation réciproquement
            bénéfique entre l’entreprise et ses clients.
        \subsection[La fidélisation de la clientèle]{\textit{La fidélisation de la clientèle}}
            \subsubsection[Définition]{\textit{Définition}}
            La fidélisation de la clientèle consiste à mettre
            en place des actions marketing et commerciales
            pour construire une relation durable avec les
            clients et les inciter à renouveler leurs achats
            dans un laps de temps plus ou moins long. \cite*{Maud2022}
            \par
            Cela peut inclure des programmes de fidélité, des offres
            spéciales, des récompenses, etc.
            L’objectif est de créer un attachement à la marque et
            d’encourager les clients à revenir.
            \par
            Les entreprises qui parviennent à bien cerner ce qu’est
            la fidélisation de la clientèle et son importance n’hésitent
            pas à mener toutes les actions possibles pour rendre les clients
            fidèles, car en capitalisant sur les clients satisfaits
            qui achètent et consomment les produits et services, les entreprises
            peuvent générer des revenus réguliers.
            \par
            % La fidélisation est considérée comme un concept
            % marketing qui touche aussi le domaine de la relation client
            Même si la fidélisation représente davantage un concept marketing,
            elle concerne également le domaine de la relation client.
            \subsubsection[Enjeux de la fidélisation]{\textit{Enjeux de la fidélisation}}
            Sur des marchés de plus en plus saturés,
            où la situation concurrentielle se durcit, il apparait que les coûts de prospection
            de nouveaux clients sont supérieurs aux coûts de conservation des clients. \cite*{Reichheld2001loyalty}
            De ce fait, une société rationnelle préfère investir pour conserver
            les clients qu’il a, plutôt que de tenter de conquérir les clients servis par
            la concurrence.
            \par
            Des études montrent qu’il existe en longue période une corrélation entre capacité
            d’une organisation à fidéliser ses clients (taux de rétention élevé) et ses
            résultats concrets (exprimés en part de marché, en rentabilité et en croissance). \cite*{Reichheld2001loyalty}
            %% A MODIFIER
            Les entreprises qui sont en
            mesure de conserver leur base clientèle et en particulier leurs « bons clients »
            sont celles qui non seulement résistent le mieux aux dépressions conjoncturelles,
            mais aussi sont les plus capables de financer leurs projets de développement.
        \subsection[Les outils d’écoute]{\textit{Les outils d’écoute}}
            \subsubsection[Identifier les besoins]{\textit{Identifier les besoins}}
            \subsubsection[Évaluer la satisfaction]{\textit{Évaluer la satisfaction}}
            \subsubsection[Définir des actions d’amélioration]{\textit{Définir des actions d’amélioration}}
            \subsubsection[Mesurer les performances clients]{\textit{Mesurer les performances clients}}
        \subsection[Le système décisionnel]{\textit{Le système décisionnel}}
        Un système d’aide à la décision (SAD), ou Decision Support System (DSS)
        en anglais, aide les utilisateurs à prendre des décisions.
        Il s’agit d’un programme qui aide les entreprises à porter des jugements
        et à déterminer des plans d’action.
        Un système d’aide à la décision examine
        de grandes quantités de données, les analyse et les organise sous forme
        de rapports complets qui facilitent la prise de décision et
        la résolution des problèmes. \cite*{SAD}
            \subsubsection[Business intelligence]{\textit{Business intelligence}}
    \section[Cadre théorique]{Cadre théorique}
    \section[Conclusion partielle]{Conclusion partielle}
        \chapter[ANALYSE CONCEPTUELLE DU SYSTÈME D’INFORMATION]{ANALYSE CONCEPTUELLE DU SYSTÈME D’INFORMATION}
    \section[Introduction partielle]{Introduction partielle}
    Dans ce chapitre, nous allons présenter l’environnement professionnel dans lequel notre
    travail se déroule.
    \par
    Premièrement nous commençons d’abord par une brève présentation de Classic Coach, puis nous introduisons
    la structure générale de son organisation avec ses différentes directions en particulier
    sa direction marketing pour qui notre projet est destiné, nous détaillons son organigramme ainsi
    que ses objectifs. 
    \par
    Et deuxièmement nous faisons une description du système d’information existant ainsi
    que du futur système d’information en ressortissant les besoins fonctionnels et non
    fonctionnels de ce dernier.
    \section[Environnement de travail]{Environnement de travail}
        \subsection[Présentation]{\textit{Présentation}}
        La société Classic Coach est une compagnie de transport routier
        par bus de biens et de personnes, en Afrique et particulièrement
        en République Démocratique du Congo.
        
        \subsection[Aperçu historique]{\textit{Aperçu historique}}
        La société Classic Coach à été créée par Monsieur AMIR RASHID ABDALLAH
        de nationalité tanzanienne, elle débute ses activités en 2007 dans le
        souci de rendre facile les déplacements de la population d’un pays à un autre, d’une ville à une autre,
        d’une citée à une autre. Rapidement la compagnie se développe et effectue
        aujourd’hui au départ du Congo, Tanzanie, Kenya, Afrique du Sud, Burundi,
        Rwanda, Zimbabwe, Zambie et autres grandes villes des pays de l’Afrique
        centrale et celui de l’ouest. 

        \subsection[Situation géographique]{Situation géographique}
        Son siege social, se situe dans la province du Haut-Katanga, dans la ville de Lubumbashi,
        dans la commune Annexe sur le boulevard M'siri au numéro 256.
    \section[Analyse du système existant]{Analyse du système existant}
        \subsection[short]{*** Procédures de ventes des billets}
        \subsection[short]{title}
        \subsection[short]{\textit{Système de fidélisation existant}}
        \subsection[short]{\textit{Système de prise de décision existant}}
        \subsection[Critique de l’existant]{\textit{Critique de l’existant}}
    \section[Futur système]{Futur système}
    \section[Conclusion partielle]{Conclusion partielle}
        \chapter[Analyse et conception du système]{ANALYSE ET CONCEPTION DU SYSTÈME}
    \section[Introduction partielle]{Introduction partielle}
    \section[short]{Analyse des besoins et conception du système}
        \subsection[Identification des besoins]{\textit{Identification des besoins}}
        \subsection[Spécification des besoins]{\textit{Spécification des besoins}}
        \subsection[La vue du processus]{\textit{La vue du processus}}
        \subsection[La vue logique]{\textit{La vue logique}}
        \subsection[Le diagramme de classe]{\textit{Le diagramme de classe}}
        \subsection[Le diagramme de navigation]{\textit{Le diagramme de navigation}}
    \section[Conclusion partielle]{Conclusion partielle}
        \chapter[MISE EN ŒUVRE]{MISE EN ŒUVRE}
    \section[Introduction partielle]{Introduction partielle}
    Ce chapitre décrira les différentes étapes par lesquelles nous
    sommes passé afin de réaliser notre application web. En premier
    lieu, notre chapitre débutera par l’environnement de réalisation
    de la solution. Ensuite, nous montrerons notre base de données qui a
    comme fondement le diagramme de modèle de domaine vu dans
    le chapitre précèdent. Enfin, nous démontrerons la manière dont
    nous avons choisi notre méthode, construit notre ..............
    \section[Environnement logiciel]{Environnement logiciel}
        \subsection[Choix des langages de développement]{Choix des langages de développement}
            \begin{itemize}
                \setlength{\itemsep}{0pt}
                \item [\ding{226}] \textbf{HTML :} c’est un langage de balisage qui permet de créer et représenter le
                contenu d’une page web et sa structure. Nous l’avons choisi pour organiser nos
                différentes pages web.
                \item [\ding{226}] \textbf{CSS :} c’est un langage de balisage utilisé pour mettre en forme le contenu d’une
                page web. Nous l’avons choisi pour donner une forme agréable aux pages web
                de notre application
                \item [\ding{226}] \textbf{Javascript :} c’est un langage de programmation des scripts utilisé dans les pages
                web interactives. Nous l’avons choisi, car il permet de rendre le contenu mis à
                jour de façon dynamique.
                \item [\ding{226}] \textbf{Python :} est un langage portable, dynamique, extensible, gratuit, qui permet (sans l’imposer) une approche
                modulaire et orienté objet de la programmation. Python est développé depuis 1989 par Guido van Rossum
                et de nombreux contributeurs bénévoles. \cite*{Swinnen2012}
            \end{itemize}
        \subsection[Choix des frameworks]{Choix des frameworks}
            \begin{itemize}
                \setlength{\itemsep}{0pt}
                \item [\ding{226}] \textbf{Bootstrap :}
                \item [\ding{226}] \textbf{Django :} est construit sur une approche très modulaire ; il implémente avec efficacité
                les notions de découplage et de réutilisabilité.
            \end{itemize}
        \subsection[Choix des outils de développement]{Choix des outils de développement}
            \begin{itemize}
                \setlength{\itemsep}{0pt}
                \item [\ding{226}] \textbf{GitHub Desktop :}
                \item [\ding{226}] \textbf{Visual Studio Code :} est un éditeur de code
                extensible développé par Microsoft. Il est un éditeur de code multiplateforme, open source et gratuit,
                supportant une multitude de langages de programmation. \cite*{Wikivsc} Cet outil a était choisi
                pour nous permettre de développer notre application web, mais aussi de rédiger ce document grâce à
                ses nombreuses extensions.
                \item [\ding{226}] \textbf{Microsoft Power BI Desktop :} est l’outil de Microsoft spécifiquement destiné à la visualisation de
                données, à la création de rapports, à l’aide au pilotage de l’entreprise, mais aussi à la
                diffusion de l’information sur différents supports ou plateformes. \cite*{Meyer2021}
            \end{itemize}
        \subsection[Choix du SGBD]{Choix du SGBD}
            \begin{itemize}
                \setlength{\itemsep}{0pt}
                \item [\ding{226}] \textbf{Microsoft SQL Server :}
            \end{itemize}
    \section[Environnement matériel]{Environnement matériel}
    \section[Captures]{Captures}
    \section[Conclusion partielle]{Conclusion partielle}
    Ce chapitre a était dédié à l’implémentation du système, où nous avons exposé
    les technologies et les outils tout au long du processus de mise en œuvre,
    en plus de détailler l’environnement matériel et les modules de l’application.
    Enfin, nous concluons ce chapitre en présentant l’application à travers quelques
    captures d’écran.
    \par
    Nous voilà arrivés pratiquement à la fin de notre travail ; il est indispensable de récapituler
    de manière condensée le contenu du présent travail de fin d’étude. Le prochain point
    conclu le travail dans sa généralité.
        \chapter*{CONCLUSION GÉNÉRALE}
\addcontentsline{toc}{chapter}{CONCLUSION GÉNÉRALE}
De nos jours, l’information est considérée comme la clé primaire
de l’économie et constitue un instrument de compétition,
c’est pourquoi les entreprises sont conscientes de la maitrise 
de l’information afin de la rendre disponible sous
la bonne forme, au moment opportun à la bonne personne qui sera
l’exploiter. C’est pour cela qu’il faut accompagner les décideurs des 
entreprises par d’outils analytiques sophistiqués pour leur
permettre de prendre des décisions pertinentes.
\par
Et donc, au terme de ce travail qui a porté sur
la \enquote{Mise en place d’un système de fidélisation de la clientèle
intégrant un module d’aide à la décision dans une entreprise
de transport de biens et de personnes} cas de la société Ngokaf Trans. 
Nous avons présenté le processus métier de Ngokaf Trans
qui était dépourvu d'un système d'information efficace, nous avons
dégagé les besoins fonctionnels pour une meilleure conception du futur système.
L’objectif poursuivi est de mettre en avant un système d'information qui
permettrai de centraliser les données et de les rendre accessible à qui de droit, 
de permettre aux clients à évaluer les services offerts par ladite entreprise et enfin
d'aider la sphère décisionnelle de savoir quand fidéliser leur segment client ou 
quand améliorer les services offerts. Le tout a était implémenté suivant la
méthode UP et la méthode GIMSI.
\par
Pour finir, nous pouvons conclure que malgré les obstacles rencontrés,
ce projet de fin d’études est une expérience amplement enrichissante
qui nous a permis d’accroitre notre savoir, tout en mettant en pratique
nos connaissances acquise durant notre cursus cela nous a permis aussi 
d’explorer de nouvelles technologies et d’outils d’analyse et de
développement, et aussi cela nous a permis de dire que c’est une
agréable expérience professionnelle qui nous a fait découvrir
le marché du travail afin de nous préparer à continuer notre chemin.
\par
La solution apportée n’est pas la seule à être valide ou optimale,
le travail reste ouvert aux différents compléments qui participeront
à son amélioration et espérons qu’il servira
de référence à tout chercheur traitant sur le même sujet que nous.
Reconnaissant sincèrement les
insuffisances qui régissent les œuvres humains, nous souhaitons la
bienvenue à toutes remarques, suggestions et conseils.

%%%%%%%%%%%%%%%%%%%%%%
%%% MOM I LOVE YOU %%%
%%%%%%%%%%%%%%%%%%%%%%
    \backmatter
        \appendix
        \printbibliography[heading=bibintoc,title={RÉFÉRENCES}]
\end{document}
