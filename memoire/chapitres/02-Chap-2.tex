\chapter[ANALYSE CONCEPTUELLE DU SYSTÈME D’INFORMATION]{ANALYSE CONCEPTUELLE DU SYSTÈME D’INFORMATION}
    \section[Introduction partielle]{Introduction partielle}
    Dans ce chapitre, nous allons présenter l’environnement professionnel dans lequel notre
    travail se déroule.
    \par
    Premièrement nous commençons d’abord par une brève présentation de Classic Coach, puis nous introduisons
    la structure générale de son organisation avec ses différentes directions en particulier
    sa direction marketing pour qui notre projet est destiné, nous détaillons son organigramme ainsi
    que ses objectifs. 
    \par
    Et deuxièmement nous faisons une description du système d’information existant ainsi
    que du futur système d’information en ressortissant les besoins fonctionnels et non
    fonctionnels de ce dernier.
    \section[Environnement de travail]{Environnement de travail}
        \subsection[Présentation]{\textit{Présentation}}
        La société Classic Coach est une compagnie de transport routier
        par bus de biens et de personnes, en Afrique et particulièrement
        en République Démocratique du Congo.

        \subsection[Aperçu historique]{\textit{Aperçu historique}}
        La société Classic Coach à été créée par Monsieur AMIR RASHID ABDALLAH
        de nationalité tanzanienne, elle débute ses activités en 2007 dans le
        souci de rendre facile les déplacements de la population d’un pays à un autre, d’une ville à une autre,
        d’une citée à une autre. Rapidement la compagnie se développe et effectue
        aujourd’hui au départ du Congo, Tanzanie, Kenya, Afrique du Sud, Burundi,
        Rwanda, Zimbabwe, Zambie et autres grandes villes des pays de l’Afrique
        centrale et celui de l’ouest. 

        \subsection[Situation géographique]{Situation géographique}
        Son siege social, se situe dans la province du Haut-Katanga, dans la ville de Lubumbashi,
        dans la commune Annexe sur le boulevard M’siri au numéro 256.        

        \subsection[Objectifs sociaux]{Objectifs sociaux}
        La société Classic Coach \acrshort{sarl} est une société commerciale,
        dont l’activité principale est :
        \par
        \begin{itemize}
            \setlength{\itemsep}{0pt}
            \item [\ding{226}] Le transport des personnes et des biens entres les différentes villes.
            \item [\ding{226}] Offrir les meilleurs services aux clients.
            \item [\ding{226}] Offrir des tarifs à bas prix permettent aux clients de voyager à travers les villes.
        \end{itemize}
        \subsection[Structure Organisationnelle]{Structure Organisationnelle}
            \subsubsection[Le Directeur Général]{Le Directeur Général}
            Le Directeur Générale planifie, dirige et supervise les activités reliées
            au transport, aux études d’élargissement, au rayonnement interne et externe
            de l’agence ainsi qu’à l’administration générale de la société.
            \par\noindent
            Le Directeur Générale de Classic Coach s’assure également que les valeurs
            institutionnelles et les exigences de performances au sein de l’entreprise sont respectées.

            \subsubsection[Le Manager]{Le Manager}
            Le Manager est le représentant numéro 1 du Directeur vu que celui-ci ne vit pas au pays,
            il a le rôle d’anticiper les risques, les tendances et les opportunités ; c’est lui décide
            et fait les choix stratégiques et tactiques. C’est lui qui recadre évalue les agents ;
            il a en même temps de rôle d’écouter, rédiger et présenter les feeds-back.
            Les problèmes cruciaux au sein de Classic Coach sont résolus par le Manager,
            la fonction de fédérer et motiver et superviser toutes les agences lui reviennent.

            \subsubsection[Le Secrétaire de Direction]{Le Secrétaire de Direction}
            Le secrétariat de direction chez Classic Coach est un bureau stratégique
            qui collabore directement avec le Manager il joue un rôle fondamental dans
            la bonne marche.
            \par\noindent
            Les compétences techniques du secrétariat de Direction chez Classic Coach
            lui permettent d’organiser et d’encadrer le travail administratif dont il a la charge.

            \subsubsection[Le Responsable des Ressources Humaines]{Le Responsable des Ressources Humaines}
            Il a comme mission :
            \par
                \begin{itemize}
                    \setlength{\itemsep}{0pt}
                    \item [\ding{226}] Établir et contrôler les paies spécifiques ;
                    \item [\ding{226}] Tenir à jour les dossiers du personnel et remplir les obligations légales ;
                    \item [\ding{226}] Organiser les élections des instances représentatives du personnel ;
                    \item [\ding{226}] Recruteur et intégrer le personnel.
                \end{itemize}
            \subsubsection[Le Contrôleur de Gestion]{Le Contrôleur de Gestion}
            Il a comme mission :
            \par
                \begin{itemize}
                    \setlength{\itemsep}{0pt}
                    \item [\ding{226}] Établir les prévisions d’activité en terme d’objectifs de budgets,
                    d’organisation et de moyen ;
                    \item [\ding{226}] Élaborer et adapter les outils d’analyse, les indicateurs et
                    procédures du contrôle de gestion ;
                    \item [\ding{226}] Identifier les écarts significatifs entre les réalisations et les prévisions ;
                    \item [\ding{226}] Mesure et analyser les écarts sous forme de statistiques, de tableaux de bord commentés et de rapport d’activités ;
                    \item [\ding{226}] Assurer la retransmission commentée des informations auprès de la direction générale.
                \end{itemize} 
            \subsubsection[Le Responsable Financier]{Le Responsable Financier}
            Il a comme mission :
            \par
                \begin{itemize}
                    \setlength{\itemsep}{0pt}
                    \item [\ding{226}] Contrôler la comptabilité de l’entreprise et la bonne
                    gestion de sa trésorerie, soit valider la rentabilité de l’entreprise ;
                    \item [\ding{226}] Développer des outils d’aide à la prise de décision ;
                    \item [\ding{226}] Valider la solvabilité de l’entreprise ;
                    \item [\ding{226}] Anticiper la stratégie de développement de l’entreprise
                    et les différents investissements et financement nécessaires.
                \end{itemize}
            \subsubsection[Le Responsable Informatique]{Le Responsable Informatique}
            Il a comme mission :
            \par
                \begin{itemize}
                    \setlength{\itemsep}{0pt}
                    \item [\ding{226}] Définir la stratégie et des objectifs en matière de développement information ;
                    \item [\ding{226}] Assurer l’organisation, le suivi et la validation des développements informatique ;
                    \item [\ding{226}] Mettre en place des projets d’évaluation en fonction des
                    besoins des utilisateurs ;
                    \item [\ding{226}] Exercer une veille sur les évolutions technologiques et
                    être force de proposition auprès de la direction ;
                    \item [\ding{226}] Définir la politique de maintenance du parc micro ;
                    \item [\ding{226}] Superviser l’achat des équipements informatiques et de logiciels ;
                \end{itemize}
        \subsection[Organigramme de l’entreprise]{Organigramme de l’entreprise}
            \begin{figure}[H]
                \centering
                \includegraphics[width=140mm]{organigramme.png}
                \caption{Organigramme de la société Classic Coach}
                \label{fig:Organigramme}
            \end{figure}
    \section[Analyse du système existant]{Analyse du système existant}
        \subsection[short]{*** Procédures de ventes des billets}
        \subsection[short]{title}
        \subsection[Système de fidélisation existant]{Système de fidélisation existant}
        \subsection[Système de prise de décision existant]{Système de prise de décision existant}
        \subsection[Critique de l’existant]{Critique de l’existant}
    \section[Futur système]{Futur système}
    \section[Conclusion partielle]{Conclusion partielle}