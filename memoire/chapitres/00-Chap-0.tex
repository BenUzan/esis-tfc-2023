\chapter*{INTRODUCTION GÉNÉRALE}
\addcontentsline{toc}{chapter}{INTRODUCTION GÉNÉRALE}
    %%%%%%%%%%%%%%%%%%%%%%%%%%%%%%%%%%%%%
    %% SECCTION A ABSOLUMENT SUPPRIMER %%
    %%%%%%%%%%%%%%%%%%%%%%%%%%%%%%%%%%%%%
    \section[Sujet]{Sujet}
    La présente étude porte sur la \enquote{Conception d’un système informatisé
    de fidélisation de la clientèle intégrant un système d’aide à la décision
    dans une entreprise de transport de biens et de personnes}.
    À ce sujet, nous avons pris le cas de la société de transport TRANS NGOKAF.
    
    \section[Contexte du sujet]{Contexte du sujet}
    La recherche scientifique anime aujourd’hui tout chercheur
    à pouvoir observer de manière
    plus particulière son environnement. Ce dernier étant
    sans cesse changeant, le chercheur se voit donc
    dans l’obligation de s’adapter aux différentes transformations
    survenant dans son environnement, ce qui n’est pas chose facile.
    Ainsi, suite à tous ces changements, le scientifique au cœur du
    développement rencontre plusieurs problèmes à résoudre en
    fonction de son domaine de recherche notamment, l’informatique,
    la médecine, l’architecture, l’économie, etc.
    \newline

    À l’exemple du domaine
    économique, la prise des décisions est devenue
    le point primordial pour une bonne gestion de
    l’entreprise. Une bonne décision engendre l’efficience,
    c’est-à-dire la réalisation des objectifs
    poursuivis tout en produisant une valeur ajoutée.
    \newline

    La plupart des entreprises du monde disposent d’une masse de données plus ou
    moins considérable. Ces informations proviennent soit de sources internes (générées par
    leurs systèmes opérationnels au fil des activités journalières), ou bien de sources externes
    (web, partenaire, etc.). Cette surabondance de données, et l’impossibilité des systèmes
    opérationnels de les exploiter à des fins d’analyse conduit, inévitablement, l’entreprise à se
    tourner vers un nouvel informatique dite décisionnelle qui met l’accent sur la
    compréhension de l’environnement de l’entreprise et l’exploitation de ces données à bon
    escient.
    \newline

    En effet, les décideurs de l’entreprise ont besoin d’avoir une meilleure vision de
    leur environnement et de son évolution, ainsi, que des informations auxquelles ils peuvent
    se fier. Cela ne peut se faire qu’en mettant en place des indicateurs de performance clairs
    et pertinents permettant la sauvegarde, l’utilisation de la mémoire de l’entreprise et offrant
    à ses décideurs la possibilité de se projeter et de se reporter à ces indicateurs pour une bonne
    prise de décision.
    \newline

    Ainsi toutes les entreprises commerciales partagent aussi
    le plus souvent un certain nombre de souhaits : gagner
    du temps, prendre du recul par rapport aux urgences, obtenir une meilleure
    stabilité de leurs recettes, mieux organiser leur travail et obtenir un meilleur
    revenu. \cite*{Barouch2010}
    \newline

    La rentabilité de la masse d’information (pouvoir transformer ce déluge
    d’information en valeur ajoutée), l’information étant un gage de compétitivité, surtout
    de nos jours, toute entreprise voulant prospérer se doit de prioriser la bonne gestion de
    cette dernière.

    \section[Problématique]{Problématique}
    La rédaction de tout travail scientifique implique au préalable une préoccupation.
    C’est dans le soucie de fidéliser le segment client de la société TRANS NGOKAF qui est une
    entreprise active dans le domaine du transport des biens et des personnes que nous
    réalisons ce travail.
    \newline
        
    De par le monde, les entreprises détenant une très grande part de marché
    se sentent le plus souvent à l’abri de toute concurrence
    dans leurs domaines d’activité grâce à la maitrise des rouages des affaires
    ou d’une quelconque technologie. \cite*{Rouviere2010} Et de ce fait
    certain d’entre elles ne s’occupent plus tant que ça de leurs relations clients.
    \newline

    C’est ainsi que nous prenons comme cas d’étude une des filiales de l’entreprise NGOKAF,
    qui exerce ses activités dans le domaine du transport de biens et de personnes, dénommée
    TRANS NGOKAF.
    \newline

    Notre étude se situe dans un environnement où la fidélisation n'est pas la principale
    préoccupation de l’entreprise, ce qui est une erreur, car c’est bien là la clé
    de la rentabilité et de la réussite de l’entreprise, car elle la rend plus performante.
    Pour cela nous essayerons d’articuler nos recherches autour des questions sui : 
    \newline

    Une question générale :
    \newline 
        \begin{itemize}
            \item [\ding{226}] Comment la société de transport TRANS NGOKAF gère-t-elle
            la relation client ?
            \newline
            \item [\ding{226}] Comment TRANS NGOKAF fait-elle pour fidéliser sa clientèle ?
            \newline
            \item [\ding{226}] Comment identifie-t-elle les clients pour une
            meilleure relation client ?
            \newline
            \item [\ding{226}] De quelle manière obtient-elle le retour client ?
        \end{itemize}
    \section[Hypothèses générales]{Hypothèses générales}
    Après avoir réalisés des recherches préliminaires, nous pouvons émettre, au regard
    de notre problématique, les hypothèses suivantes : 
    \newline
    \begin{itemize}
        \item [\ding{226}] Une application web sera réalisée, ce qui permettra de collecter
        les informations de la clientèle et leurs retours concernant les prestations de l’entreprise.
        \newline
        \item [\ding{226}] Afin d’effectuer du marketing ciblé sur les clients, un outil d’aide à
        la décision sera intégrer à l’application.
        \newline
        \item [\ding{226}] Un moyen de fidélisation par campagne promotionnelle sera réaliser.
        \newline
    \end{itemize}
    
    \section[Choix et interet du sujet]{Choix et intérêts du sujet}
        \subsection[Choix du sujet]{\textit{Choix du sujet}}
        Dans le cadre général, le choix porté sur ce sujet a été motivé par le souci d’aider
        la société TRANS NGOKAF de pouvoir fidéliser plus aisément le segment client à laide d’un
        système intégrant un système d’aide à la décision.
        \subsection[Interet du sujet]{\textit{Intérêts du sujet}}
        L’intérêt du sujet se réfère à l’importance et à la pertinence du sujet choisi pour notre recherche.
        Un sujet intéressant est celui qui est pertinent pour les autres parties impliquées dans nos recherches.
        Il doit être capable de susciter l’intérêt et de motiver les recherches et la rédaction. C’est ainsi
        que nous nous en retiendrons trois :
         \newline 
            \begin{itemize}
                \item [\ding{226}] Intérêt personnel : en tant qu’ingénieur en management 
                des systèmes d’information, nous serons heureux d’apporter une solution
                qui aidera les entreprises de transport de biens et de personnes de fidéliser
                le segment client.
                \newline

                \item [\ding{226}] Intérêt sociétal : la réalisation de ce système
                informatisé de fidélisation de la clientèle permettra à TRANS NGOKAF
                de conserver sa part de marché actuel tout en la développant. 
                %%%%%%%%%%%%%%%%%%%%%%%%%%%%%%%%%%%%%%%%%%%
                %% SOLUTION POUR LES GENERATIONS FUTURES %%
                %%%%%%%%%%%%%%%%%%%%%%%%%%%%%%%%%%%%%%%%%%%
                %% et dans le meilleur des cas mettre en place une stratégie
                %% de prospection pour accroitre sa clientèle.
                Selon Bain \& Cie : \enquote{5 \% d’augmentation du taux de
                rétention sur les meilleurs clients peut générer entre 25 et 55 \%
                d’augmentation des bénéfices d’une entreprise.} \cite*{Siecdigi}
                \newline

                \item [\ding{226}] Intérêt scientifique : dans le système LMD
                \footnote[1]{LMD : Licence-Master-Doctorat} tout étudiant
                a le devoir, à la fin de son cycle d’étude, d’élaborer un travail qui
                sanctionne son parcours. Le bien fonder du travail de fin de cycle
                est de donner l’occasion à l’étudiant de prouver la maitrise et la bonne acquisition
                des notions apprises tout au long de son parcours dans une filière données.
            \end{itemize}
    \section[Démarches méthodologiques]{Démarches méthodologiques}
        \subsection[Méthodes]{\textit{Méthodes}}
        Une méthode est une manière de conduire sa pensée, d’établir ou de démontrer une
        vérité suivant certains principes et avec un certain ordre.
        \newline

        Dans le cadre de notre travail, nous utiliserons la méthode UP (Unified Process).
        Le Processus Unifié est un processus de développement logiciel \enquote{itératif et incrémental,
        centré sur l’architecture, conduit par les cas d’utilisation et piloté par les risques.} \cite{Roques2008}
        \newline

        UML se définit comme un langage de modélisation graphique et textuel destiné à
        comprendre et décrire des besoins, spécifier et documenter des systèmes, esquisser des
        architectures logicielles, concevoir des solutions et communiquer des points de vue. \cite{RoqVall2007}
        \newline
        
        UML est le moyen graphique de garantir que \enquote{ce qui se conçoit et se programme
        bien s’énonce clairement.\footnote[2]{Hugues \textsc{Bersisni}, \textit{La programmation orientée objet}, 7e édition, Eyrolles, Paris, 2017, p. 222}}
        \subsection[Techniques]{\textit{Techniques}}
        Les techniques sont des mécanismes qui nous permettent de réaliser nos recherches
        scientifiques. Pour rendre notre travail facile à réaliser nous avons pris comme techniques : 
        \newline
        \begin{itemize}
            \item [\ding{226}] L’interview : pour comprendre de façon simple notre travail, nous avons fait des
            descentes sur terrains, nous avons eu à interroger les techniciens et le manager
            informatiques, les professeurs informatiques, les collègues, les ainés scientifiques
            et ceux qui s’intéressent beaucoup plus à l’outil informatique.
            \newline
            \item [\ding{226}] L’observation : nous nous sommes données beaucoup du temps à observer plus
            attentivement, pour comprendre de manière précise comment ça se passe exactement sur un ordinateur fonctionnel.
            \newline
            \item [\ding{226}] La documentation : la lecture est le moyen efficace de voyager dans le monde
            de la connaissance. Comme un chercheur, pour parvenir à résoudre un problème
            dans la société, il nous faut une lecture consistante. Nous avons eu à utiliser les
            livres scientifiques, nous nous sommes rendus dans les bibliothèques, nous avons
            consulté des documents sur l’internet, nous avons utilisé différents médias afin de
            réunir toutes les informations dont nous avons besoin.
        \end{itemize}
    \section[Etat de l'art]{État de l’art}
    Ce travail se basant sur des informations et une structure d’entreprise particulière, nous ne
    pouvons pas déclarer que ce dernier est une recherche originale, car d’autre chercheur ont eu
    à traiter du sujet relativement similaire à celui-ci.
    \newline

    Nous avons notamment BUYAMABA SUZE Ange, dans le cadre de son travail de fin de cycle
    qui parlait de : \enquote{MISE EN PLACE D’UN SYSTÈME INFORMATISE DE FIDÉLISATION DES CLIENTS BASE SUR LES
    DONNÉES D’UN SERVICE TRAITEUR}. Ce travail base sur la conception d’un système informatisé de fidélisation des clients
    à aider la maison Excellence d’obtenir les retours clients dans la perspective d’une amélioration des services. \cite{Buyamba2017}
    \newline

    Cependant, il est impossible à la maison Excellence de savoir, grâce à des analyses et des visualisations, qui
    sont les meilleurs clients. Et sans oublier que sa recherche était concentré sur les services traiteurs. 
    \newline

    Mais en ce qui nous concerne nous rejoignions le chercheur cité ci-haut, dans le sens où nous traitons de manière
    générale d’une application de fidélisation, mais intégrant un système d’aide à la décision.
    % Divergence et conc=vergence -- demarcation
    \section[Délimitation du travail]{Délimitation du travail}
        Nos recherches ont été effectuer chez TRANS NGOKAF qui est une entreprise de transport
        située dans la ville de Lubumbashi, dans la province du Haut-Katanga en République Démocratique du Congo.
        \newline
        Notre Travail couvrira l’année 2023.
    \section[Subdivision du travail]{Subdivision du travail}
    Outre l’introduction générale et la conclusion générale, notre travail est subdivisé
    en trois chapitres :
    \newline
        \begin{itemize}
            \item [\ding{226}] Le premier chapitre : \enquote{cadre conceptuel et théorique}, présente une vue
            d’ensembles des concepts de base du sujet et quelques théories sur la méthodologie
            utilisée du travail.
            \newline
            \item [\ding{226}] Le deuxième chapitre : \enquote{analyse conceptuelle du système d’information}, 
            décrit l’analyse de l’architecture métier dans lequel nous allons présenter un système existant
            et le processus de l’organisation du travail ainsi que le futur système.
            \newline       
            \item [\ding{226}] Le troisième chapitre : \enquote{analyse et conception du système}, s’appuie sur la
            modélisation de notre système métier et est basé sur l’explication de différentes
            technologies utilisées pour l’implémentation ou développement de notre solution.            
        \end{itemize} 
    \section[Outils logiciels et équipements utilisés]{Outils logiciels et équipements utilisés}
    %%%%%%%%%%%%%
    %% A VENIR %%
    %%%%%%%%%%%%%
    Après avoir présenté le contexte et les enjeux de notre recherche, nous allons maintenant nous pencher sur
    les différents concepts, théories et méthodes utilisées.