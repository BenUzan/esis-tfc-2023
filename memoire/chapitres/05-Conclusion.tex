\chapter*{CONCLUSION GÉNÉRALE}
\addcontentsline{toc}{chapter}{CONCLUSION GÉNÉRALE}
De nos jours, l’information est considérée comme la clé primaire
de l’économie et constitue un instrument de compétition,
c’est pourquoi les entreprises sont conscientes de la maitrise 
de l’information afin de la rendre disponible sous
la bonne forme, au moment opportun à la bonne personne qui sera
l’exploiter. C’est pour cela qu’il faut accompagner les décideurs des 
entreprises par d’outils analytiques sophistiqués pour leur
permettre de prendre des décisions pertinentes.
\par
Et donc, au terme de ce travail qui a porté sur
la \enquote{Mise en place d’un système de fidélisation de la clientèle
intégrant un module d’aide à la décision dans une entreprise
de transport de biens et de personnes} cas de la société Ngokaf Trans. 
Nous avons présenté le processus métier de Ngokaf Trans
qui était dépourvu d'un système d'information efficace, nous avons
dégagé les besoins fonctionnels pour une meilleure conception du futur système.
L’objectif poursuivi est de mettre en avant un système d'information qui
permettrai de centraliser les données et de les rendre accessible à qui de droit, 
de permettre aux clients à évaluer les services offerts par ladite entreprise et enfin
d'aider la sphère décisionnelle de savoir quand fidéliser leur segment client ou 
quand améliorer les services offerts. Le tout a était implémenté suivant la
méthode UP et la méthode GIMSI.
\par
Pour finir, nous pouvons conclure que malgré les obstacles rencontrés,
ce projet de fin d’études est une expérience amplement enrichissante
qui nous a permis d’accroitre notre savoir, tout en mettant en pratique
nos connaissances acquise durant notre cursus cela nous a permis aussi 
d’explorer de nouvelles technologies et d’outils d’analyse et de
développement, et aussi cela nous a permis de dire que c’est une
agréable expérience professionnelle qui nous a fait découvrir
le marché du travail afin de nous préparer à continuer notre chemin.
\par
La solution apportée n’est pas la seule à être valide ou optimale,
le travail reste ouvert aux différents compléments qui participeront
à son amélioration et espérons qu’il servira
de référence à tout chercheur traitant sur le même sujet que nous.
Reconnaissant sincèrement les
insuffisances qui régissent les œuvres humains, nous souhaitons la
bienvenue à toutes remarques, suggestions et conseils.

%%%%%%%%%%%%%%%%%%%%%%
%%% MOM I LOVE YOU %%%
%%%%%%%%%%%%%%%%%%%%%%