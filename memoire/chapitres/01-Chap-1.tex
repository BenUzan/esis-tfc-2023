\chapter[CADRE CONCEPTUEL ET THÉORIQUE]{\centering CADRE CONCEPTUEL ET THÉORIQUE}    
    \section{Introduction partielle}
    Actuellement, l’informatique est devenue l’outil privilégié de l’information dans
    différentes entreprises. D’où sa nécessité s’impose dans tous les domaines de gestion et la
    maitrise de l’outil informatique qui constitue la garantie d’une bonne gestion, d’un bon
    fonctionnement et donc une bonne rentabilité de cette dernière.
    \section[Cadre Conceptuel]{Cadre conceptuel}
    Le cadre conceptuel met en relation les concepts fondamentaux du travail de
    recherche. Dans cette section, nous allons aborder les théories existantes relatives à notre
    domaine d’étude comprenant trois aspects à savoir : 
    \newline
        \begin{enumerate}
            \item La gestion de la relation client
            \item La fidélisation de la clientèle
            \item Le système décisionnel
        \end{enumerate} 
        \subsection[La gestion de la relation client]{\textit{La gestion de la relation client}}
        La gestion de la relation client (CRM, customer relationship management en anglais) est
        la capacité à bâtir une relation profitable sur le long terme avec les meilleurs clients
        en capitalisant sur l’ensemble des points de contacts par une allocation optimale
        des ressources. \cite*{Lefebure2005}
        \newline
        
        L’objectif de la gestion de la relation client est de créer et d’entretenir
        une relation réciproquement bénéfique entre l’entreprise et ses clients
        \subsection[La fidélisation de la clientèle]{\textit{La fidélisation de la clientèle}}
        La fidélisation de la clientèle consiste à mettre en place des actions marketing et commerciales
        pour construire une relation durable avec les clients et les inciter à renouveler
        leurs achats dans un laps de temps plus ou moins long. \cite*{Maud2022}
        Cela peut inclure des programmes de fidélité, des offres spéciales, des récompenses, etc.
        L’objectif est de créer un attachement à la marque et d’encourager les clients à revenir.
        \newline

        Les entreprises qui parviennent à bien cerner ce qu’est la fidélisation
        de la clientèle et son importance n’hésitent pas à mener toutes les actions
        possibles pour rendre les clients fidèles, car en capitalisant sur les clients satisfaits
        qui achètent et consomment les produits et services, les entreprises peuvent générer des revenus réguliers.
        \newline
        % La fidélisation est considérée comme un concept marketing qui touche aussi le domaine de la relation client

        Même si la fidélisation représente davantage un concept marketing,
        elle concerne également le domaine de la relation client.
        \subsection[Le système décisionnel]{\textit{Le système décisionnel}}
        Un système d’aide à la décision (SAD), ou Decision Support System (DSS)
        en anglais, aide les utilisateurs à prendre des décisions.
        Il s’agit d’un programme qui aide les entreprises à porter des jugements
        et à déterminer des plans d’action.
        Un système d’aide à la décision examine
        de grandes quantités de données, les analyse et les organise sous forme
        de rapports complets qui facilitent la prise de décision et
        la résolution des problèmes. \cite*{SAD}
            \subsubsection[Business intelligence]{\textit{Business intelligence}}
    \section[Cadre théorique]{Cadre théorique}
    \section[Conclusion partielle]{Conclusion partielle}