\chapter[CADRE CONCEPTUEL ET THÉORIQUE]{CADRE CONCEPTUEL ET THÉORIQUE}
    \section{Introduction partielle}
    Actuellement, l’informatique est devenue l’outil privilégié de l’information dans
    différentes entreprises. D’où sa nécessité s’impose dans tous les domaines de gestion et la
    maitrise de l’outil informatique qui constitue la garantie d’une bonne gestion, d’un bon
    fonctionnement et donc une bonne rentabilité de cette dernière.
    \section[Cadre Conceptuel]{Cadre conceptuel}
    Le cadre conceptuel met en relation les concepts fondamentaux du travail de
    recherche. Dans cette section nous allons aborder les théories existantes relatives à notre
    domaine d’étude comprenant deux aspects à savoir : En amont nous parleront de
    \section[Cadre théorique]{Cadre théorique}