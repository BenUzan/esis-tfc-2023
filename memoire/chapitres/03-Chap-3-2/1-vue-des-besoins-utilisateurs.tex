\subsection[La vue des besoins utilisateurs]{La vue des besoins utilisateurs}    
    \subsubsection[Identification des acteurs]{Identification des acteurs}
    Voici les acteurs identifiés dans le cadre de notre
    travail :
    \par
        \begin{itemize}
            \setlength{\itemsep}{0pt}
            \item [\ding{226}] \textbf{L’internaute} : est celui qui visite
            notre plateforme sans y être inscrit.
            \item [\ding{226}] \textbf{Le client} : est celui qui a créé un compte
            utilisateur dans le système.
            \item [\ding{226}] \textbf{L’administrateur} : il est chargé de
            monitorer le système afin d’éviter tout débordement des utilisateurs,
            et voir tous les mauvais fonctionnements du système et les régler
            à distance.
            \item [\ding{226}] \textbf{Le manager} : est celui qui utilise le système
            pour visualiser le tableau de bord et consulter le rapport.
        \end{itemize}
    \subsubsection[Identification des cas d’utilisation]{Identification des cas d’utilisation}
    Les cas d’utilisations ou fonctionnalités de notre
    système sont les suivants :
    \par
        \begin{itemize}
            \setlength{\itemsep}{0pt}
            \item [\ding{226}] Créer compte
            \item [\ding{226}] S’authentifier
            \item [\ding{226}] Consulter avis
            \item [\ding{226}] Afficher notification
            \item [\ding{226}] Scanner jeton
            \item [\ding{226}] Donner avis
            \item [\ding{226}] Établir rapport
            \item [\ding{226}] Vendre billet
            \item [\ding{226}] Gérer compte
            \item [\ding{226}] Ajouter évaluation
            \item [\ding{226}] Génère tableau de bord
            \item [\ding{226}] Consulter tableau de bord
            \item [\ding{226}] Consulter rapport
        \end{itemize}
    Chacun des cas figurant dans la liste présenter
    ci-haut, seras expliquer plus en détaille dans
    sa description textuelle.

%%%%%%%%%%%%%%%%%%%%%
%% Page en paysage %%
%%%%%%%%%%%%%%%%%%%%%
\uselandscape
    \subsubsection[Diagrammes de cas d’utilisation]{Diagrammes de cas d’utilisation}
    Après avoir eu à identifier les acteurs et leurs cas d’utilisation
    voici comment se présente le diagramme de cas d’utilisation :
        \begin{figure}[H]
            \centering
            \includegraphics[width=165mm]{images/dcu_systeme.png}
            \caption{Diagramme de cas d’utilisation}
            \label{fig:dcu}
        \end{figure}

%%%%%%%%%%%%%%%%%%%%%%
%% Page en portrait %%
%%%%%%%%%%%%%%%%%%%%%%        
\useportrait
    \subsubsection[Description textuelle]{Description textuelle}
        \paragraph[Créer compte]{Créer compte}
            \begin{longtable}{p{4cm} p{9cm}}
                \caption{Description textuelle du cas d’utilisation Créer compte}
                \label{table:usecaseCreeCompte}
                \\\hline\hline
                    \textbf{Cas d’utilisation} & \textbf{Créer compte}
                \\\hline\hline
                        \textbf{Acteur principal} & L’\textsc{internaute}
                    \\
                        \textbf{Acteur secondaire} & -
                    \\
                        \textbf{Objectifs} & L’\textsc{internaute} veut être 
                        enregistré dans le système.
                    \\
                        \textbf{Préconditions} & -
                    \\
                        \textbf{Postcondition} & Un compte a était créé.
                    \\
                        \textbf{Scénario nominal} &
                            \begin{enumerate}[leftmargin=*]
                                \item L’\textsc{internaute} clique sur créer compte et le
                                système affiche un formulaire.
                                \item L’\textsc{internaute} saisi ses identifiants à savoir
                                (nom, post-nom, date et année de
                                naissance, âge et genre) et coordonnées (téléphone, adresse physique
                                et adresse électronique).
                                \item L’\textsc{internaute} confirme en cliquant sur créer.
                                \item Le système enregistre les nouvelles informations
                                dans la base de données et le
                                système répond avec un message de succès.
                            \end{enumerate}
                    \\
                        \textbf{Scénario alternatif} &
                            \begin{enumerate}[leftmargin=*]
                                \item À partir de l’étape numéro 3 du scenario nominal
                                le système vérifie et ne valide pas le formulaire
                                \item Retour à l’étape numéro 1 du scénario principal
                            \end{enumerate}
                \\\bottomrule
            \end{longtable}
            %\pagebreak

        \paragraph[S’authentifier]{S’authentifier}
            \begin{longtable}{p{4cm} p{9cm}}
                \caption{Description textuelle du cas d’utilisation S’authentifier}
                \label{table:usecaseSauth}
                \\\hline\hline
                    \textbf{Cas d’utilisation} & \textbf{S’authentifier}
                \\\hline\hline
                        \textbf{Acteur principal} & L’\textsc{internaute}, le \textsc{client},
                        le \textsc{guichetier}, le \textsc{manager} et l’\textsc{administrateur}
                    \\
                        \textbf{Acteur secondaire} & -
                    \\
                        \textbf{Objectifs} & L’utilisateur veut accéder au système.
                    \\
                        \textbf{Préconditions} & -
                    \\
                    \textbf{Scénario nominal} &
                        \begin{enumerate}[leftmargin=*]
                            \item L’utilisateur saisis les informations de connexion à savoir (le mot de
                            passe et l’identifient).
                            \item Il confirme en cliquant sur le bouton se connecter.
                            \item Le système vérifie les informations saisies et crée la session.
                        \end{enumerate}
                    \\
                    \textbf{Scénario alternatif} &
                    \begin{enumerate}[leftmargin=*]
                            \item mm
                            \item kk
                        \end{enumerate}
                    \\
                    \textbf{Postcondition}
                \\\bottomrule
            \end{longtable}

        \paragraph[Consulter avis]{Consulter avis}
            \begin{longtable}{p{4cm} p{9cm}}
                \caption{Description textuelle du cas d’utilisation Consulter avis}
                \label{table:usecaseConsulterAvis}
                \\\hline\hline
                    \textbf{Cas d’utilisation} & \textbf{Consulter avis}
                \\\hline\hline
                        \textbf{Acteur principal} & aa
                    \\
                        \textbf{Acteur secondaire} & bb
                    \\
                        \textbf{Objectifs} & dd
                    \\
                        \textbf{Préconditions} & dd
                    \\
                    \textbf{Scénario nominal} &
                        \begin{enumerate}[leftmargin=*]
                            \item dd
                            \item ddd
                        \end{enumerate}
                    \\
                    \textbf{Scénario alternatif} &
                        \begin{enumerate}[leftmargin=*]
                            \item mm
                            \item kk
                        \end{enumerate}
                    \\
                    \textbf{Postcondition}
                \\\bottomrule
            \end{longtable}

        \paragraph[Afficher notification]{Afficher notification}
        \begin{longtable}{p{4cm} p{9cm}}
            \caption{Description textuelle du cas d’utilisation Afficher notification}
            \label{table:usecaseAfficherNotification}
            \\\hline\hline
                \textbf{Cas d’utilisation} & \textbf{Afficher notification}
            \\\hline\hline
                    \textbf{Acteur principal} & aa
                \\
                    \textbf{Acteur secondaire} & bb
                \\
                    \textbf{Objectifs} & dd
                \\
                    \textbf{Préconditions} & dd
                \\
                \textbf{Scénario nominal} &
                    \begin{enumerate}[leftmargin=*]
                        \item dd
                        \item ddd
                    \end{enumerate}
                \\
                \textbf{Scénario alternatif} &
                    \begin{enumerate}[leftmargin=*]
                        \item mm
                        \item kk
                    \end{enumerate}
                \\
                \textbf{Postcondition}
            \\\bottomrule
        \end{longtable}

        \paragraph[Scanner jeton]{Scanner jeton}
        \begin{longtable}{p{4cm} p{9cm}}
            \caption{Description textuelle du cas d’utilisation Scanner jeton}
            \label{table:usecaseScannerJ}
            \\\hline\hline
                \textbf{Cas d’utilisation} & \textbf{Scanner jeton}
            \\\hline\hline
                    \textbf{Acteur principal} & aa
                \\
                    \textbf{Acteur secondaire} & bb
                \\
                    \textbf{Objectifs} & dd
                \\
                    \textbf{Préconditions} & dd
                \\
                \textbf{Scénario nominal} &
                    \begin{enumerate}[leftmargin=*]
                        \item dd
                        \item ddd
                    \end{enumerate}
                \\
                \textbf{Scénario alternatif} &
                    \begin{enumerate}[leftmargin=*]
                        \item mm
                        \item kk
                    \end{enumerate}
                \\
                \textbf{Postcondition}
            \\\bottomrule
        \end{longtable}

        \paragraph[Donner avis]{Donner avis}
        \begin{longtable}{p{4cm} p{9cm}}
            \caption{Description textuelle du cas d’utilisation Donner avis}
            \label{table:usecaseDonnerAvis}
            \\\hline\hline
                \textbf{Cas d’utilisation} & \textbf{Donner avis}
            \\\hline\hline
                    \textbf{Acteur principal} & aa
                \\
                    \textbf{Acteur secondaire} & bb
                \\
                    \textbf{Objectifs} & dd
                \\
                    \textbf{Préconditions} & dd
                \\
                \textbf{Scénario nominal} &
                    \begin{enumerate}[leftmargin=*]
                        \item dd
                        \item ddd
                    \end{enumerate}
                \\
                \textbf{Scénario alternatif} &
                    \begin{enumerate}[leftmargin=*]
                        \item mm
                        \item kk
                    \end{enumerate}
                \\
                \textbf{Postcondition}
            \\\bottomrule
        \end{longtable}

        \paragraph[Établir rapport]{Établir rapport}
        \begin{longtable}{p{4cm} p{9cm}}
            \caption{Description textuelle du cas d’utilisation Établir rapport}
            \label{table:usecaseEtablirCompte}
            \\\hline\hline
                \textbf{Cas d’utilisation} & \textbf{Établir rapport}
            \\\hline\hline
                    \textbf{Acteur principal} & aa
                \\
                    \textbf{Acteur secondaire} & bb
                \\
                    \textbf{Objectifs} & dd
                \\
                    \textbf{Préconditions} & dd
                \\
                \textbf{Scénario nominal} &
                    \begin{enumerate}[leftmargin=*]
                        \item dd
                        \item ddd
                    \end{enumerate}
                \\
                \textbf{Scénario alternatif} &
                    \begin{enumerate}[leftmargin=*]
                        \item mm
                        \item kk
                    \end{enumerate}
                \\
                \textbf{Postcondition}
            \\\bottomrule
        \end{longtable}

        \paragraph[Vendre billet]{Vendre billet}
        \begin{longtable}{p{4cm} p{9cm}}
            \caption{Description textuelle du cas d’utilisation Vendre billet}
            \label{table:usecaseVendreBill}
            \\\hline\hline
                \textbf{Cas d’utilisation} & \textbf{Vendre billet}
            \\\hline\hline
                    \textbf{Acteur principal} & aa
                \\
                    \textbf{Acteur secondaire} & bb
                \\
                    \textbf{Objectifs} & dd
                \\
                    \textbf{Préconditions} & dd
                \\
                \textbf{Scénario nominal} &
                    \begin{enumerate}[leftmargin=*]
                        \item dd
                        \item ddd
                    \end{enumerate}
                \\
                \textbf{Scénario alternatif} &
                    \begin{enumerate}[leftmargin=*]
                        \item mm
                        \item kk
                    \end{enumerate}
                \\
                \textbf{Postcondition}
            \\\bottomrule
        \end{longtable}

        \paragraph[Gérer compte]{Gérer compte}
        \begin{longtable}{p{4cm} p{9cm}}
            \caption{Description textuelle du cas d’utilisation Gérer compte}
            \label{table:usecaseGereCompte}
            \\\hline\hline
                \textbf{Cas d’utilisation} & \textbf{Gérer compte}
            \\\hline\hline
                    \textbf{Acteur principal} & aa
                \\
                    \textbf{Acteur secondaire} & bb
                \\
                    \textbf{Objectifs} & dd
                \\
                    \textbf{Préconditions} & dd
                \\
                \textbf{Scénario nominal} &
                    \begin{enumerate}[leftmargin=*]
                        \item dd
                        \item ddd
                    \end{enumerate}
                \\
                \textbf{Scénario alternatif} &
                    \begin{enumerate}[leftmargin=*]
                        \item mm
                        \item kk
                    \end{enumerate}
                \\
                \textbf{Postcondition}
            \\\bottomrule
        \end{longtable}

        \paragraph[Ajouter évaluation]{Ajouter évaluation}
        \begin{longtable}{p{4cm} p{9cm}}
            \caption{Description textuelle du cas d’utilisation Ajouter évaluation}
            \label{table:usecaseAjouterEval}
            \\\hline\hline
                \textbf{Cas d’utilisation} & \textbf{Ajouter évaluation}
            \\\hline\hline
                    \textbf{Acteur principal} & aa
                \\
                    \textbf{Acteur secondaire} & bb
                \\
                    \textbf{Objectifs} & dd
                \\
                    \textbf{Préconditions} & dd
                \\
                \textbf{Scénario nominal} &
                    \begin{enumerate}[leftmargin=*]
                        \item dd
                        \item ddd
                    \end{enumerate}
                \\
                \textbf{Scénario alternatif} &
                    \begin{enumerate}[leftmargin=*]
                        \item mm
                        \item kk
                    \end{enumerate}
                \\
                \textbf{Postcondition}
            \\\bottomrule
        \end{longtable}

        \paragraph[Générer tableau de bord]{Générer tableau de bord}
        \begin{longtable}{p{4cm} p{9cm}}
            \caption{Description textuelle du cas d’utilisation Générer tableau de bord}
            \label{table:usecaseDashboard}
            \\\hline\hline
                \textbf{Cas d’utilisation} & \textbf{Générer tableau de bord}
            \\\hline\hline
                    \textbf{Acteur principal} & aa
                \\
                    \textbf{Acteur secondaire} & bb
                \\
                    \textbf{Objectifs} & dd
                \\
                    \textbf{Préconditions} & dd
                \\
                \textbf{Scénario nominal} &
                    \begin{enumerate}[leftmargin=*]
                        \item dd
                        \item ddd
                    \end{enumerate}
                \\
                \textbf{Scénario alternatif} &
                    \begin{enumerate}[leftmargin=*]
                        \item mm
                        \item kk
                    \end{enumerate}
                \\
                \textbf{Postcondition}
            \\\bottomrule
        \end{longtable}

        \paragraph[Consulter tableau de bord]{Consulter tableau de bord}
        \begin{longtable}{p{4cm} p{9cm}}
            \caption{Description textuelle du cas d’utilisation Consulter tableau de bord}
            \label{table:usecaseConsDash}
            \\\hline\hline
                \textbf{Cas d’utilisation} & \textbf{Consulter tableau de bord}
            \\\hline\hline
                    \textbf{Acteur principal} & aa
                \\
                    \textbf{Acteur secondaire} & bb
                \\
                    \textbf{Objectifs} & dd
                \\
                    \textbf{Préconditions} & dd
                \\
                \textbf{Scénario nominal} &
                    \begin{enumerate}[leftmargin=*]
                        \item dd
                        \item ddd
                    \end{enumerate}
                \\
                \textbf{Scénario alternatif} &
                    \begin{enumerate}[leftmargin=*]
                        \item mm
                        \item kk
                    \end{enumerate}
                \\
                \textbf{Postcondition}
            \\\bottomrule
        \end{longtable}

        \paragraph[Consulter rapport]{Consulter rapport}
        \begin{longtable}{p{4cm} p{9cm}}
            \caption{Description textuelle du cas d’utilisation Consulter rapport}
            \label{table:usecaseConsulterRapport}
            \\\hline\hline
                \textbf{Cas d’utilisation} & \textbf{Consulter rapport}
            \\\hline\hline
                    \textbf{Acteur principal} & aa
                \\
                    \textbf{Acteur secondaire} & bb
                \\
                    \textbf{Objectifs} & dd
                \\
                    \textbf{Préconditions} & dd
                \\
                \textbf{Scénario nominal} &
                    \begin{enumerate}[leftmargin=*]
                        \item dd
                        \item ddd
                    \end{enumerate}
                \\
                \textbf{Scénario alternatif} &
                    \begin{enumerate}[leftmargin=*]
                        \item mm
                        \item kk
                    \end{enumerate}
                \\
                \textbf{Postcondition}
            \\\bottomrule
        \end{longtable}