\section[Identification et représentation des besoins]{Identification et représentation des besoins}    
    \subsection[Identification des acteurs]{Identification des acteurs}
    Voici les acteurs identifiés dans le cadre de notre
    travail :
    \par
        \begin{itemize}
            \setlength{\itemsep}{0pt}
            \item [\ding{226}] \textbf{L’internaute} : est celui qui visite
            notre plateforme sans y être inscrit.
            \item [\ding{226}] \textbf{Le client} : est celui qui a créé un compte
            utilisateur dans le système.
            \item [\ding{226}] \textbf{L’administrateur} : il est chargé de
            monitorer le système afin d’éviter tout débordement des utilisateurs,
            et voir tous les mauvais fonctionnements du système et les régler
            à distance.
            \item [\ding{226}] \textbf{Le manager} : est celui qui utilise le système
            pour visualiser le tableau de bord et consulter le rapport.
        \end{itemize}
    \subsection[Identification des cas d’utilisation]{Identification des cas d’utilisation}
    Les cas d’utilisations ou fonctionnalités de notre
    système sont les suivants :
    \par
        \begin{itemize}
            \setlength{\itemsep}{0pt}
            \item [\ding{226}] Créer compte
            \item [\ding{226}] S’authentifier
            \item [\ding{226}] Afficher avis
            \item [\ding{226}] Afficher notification
            \item [\ding{226}] Valider jeton
            \item [\ding{226}] Donner avis
            \item [\ding{226}] Établir rapport
            \item [\ding{226}] Vendre billet
            \item [\ding{226}] Gérer compte
            \item [\ding{226}] Ajouter évaluation
            \item [\ding{226}] Génère tableau de bord
            \item [\ding{226}] Consulter tableau de bord
            \item [\ding{226}] Consulter rapport
        \end{itemize}
    Chacun des cas figurant dans la liste présenter
    ci-haut, seras expliquer plus en détaille dans
    sa description textuelle.
\pagebreak
    \subsection[Diagrammes de cas d’utilisation]{Diagrammes de cas d’utilisation}
    Après avoir eu à identifier les acteurs et leurs cas d’utilisation
    voici comment se présente le diagramme de cas d’utilisation :
        \begin{figure}[H]
            \centering
            \includegraphics[width=150mm]{images/dcu_systeme.png}
            \caption{Diagramme de cas d’utilisation}
            \label{fig:dcu}
        \end{figure}
\pagebreak