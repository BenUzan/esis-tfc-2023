\chapter*{REMERCIEMENTS}
\addcontentsline{toc}{chapter}{REMERCIEMENTS}
    Ce travail qui coiffe ainsi la fin de notre premier cycle est le
    résultat de multiples efforts tout au long de notre formation conformément
    au programme académique. Ce produit scientifique tel qu’élaboré ne présente
    en aucune manière le fruit de nos efforts personnels, elle est l’émanation
    des efforts conjugués de plusieurs personnes d’amour et de bonne volonté
    sans lesquelles il nous serait impossible de tous les représenter.
    \par
    C’est ainsi qu’au seuil de ce travail, de labeur, de persévérance,
    de courage qu’il soit permis d’adresser nos sincères et vifs remerciements 
    au professeur André LISONGOMI BATIPONDA pour ses conseils, sa disponibilité et pour avoir également accepté
    de diriger ce travail de main de maitre malgré ses multiples occupations.
    \par
    Nos remerciements s’adressent au corps administratif et professoral de l’école supérieure d’informatique
    Salama (ESIS) particulièrement à nos coordinateurs de filière madame Allegra NZEBA et monsieur Deoel MWANAKAHAMBO. %A REVOIR
    \par
    À tous les professeurs, intervenants et toutes les
    personnes qui par leurs paroles, leurs écrits, leurs conseils et leurs critiques ont guidé mes réflexions
    et ont accepté à me rencontrer et répondre à mes questions durant mes recherches. 
    \par
    À nos amis et collègues de la grande famille Génie Logiciel
    pour leur collaboration, leur solidarité et leur présence tout au long de notre
    parcours, dans les bons comme dans les mauvais moments.
    \par
    À l’Éternel Dieu Tout-Puissant pour tous ses bienfaits durant mon parcours.
    \par
    Trouvez ici nos sincères remerciements et que Dieu bénisse tous ceux qui m’ont aidé durant mes études supérieures et
    universitaires.